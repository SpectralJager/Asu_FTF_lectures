\documentclass[a4paper, 12pt]{article}

\usepackage{graphicx}
\usepackage{xcolor}
\usepackage{mdframed}
\usepackage { amsmath , amssymb , amsthm }
\usepackage[T2A]{fontenc}
\usepackage[utf8]{inputenc}
\usepackage[english,russian]{babel}

\graphicspath{{img/}}
\DeclareGraphicsExtensions{.pdf,.png,.jpg}


\title{Инженерная графика}
\author{Щербинин В.В}
\date{\today}

\begin{document}
\maketitle

\section*{Введение}

ЕСКД -- единая система конструкторской документации (устанавливает взаимосвязь правил по оформлению, конструированию, обращинию конструкторской документации)\\
"+":\\
1. Возможность взаимообмена конструкторской документации между предприятиями.\\
2. Стабилизация комплектности, исключающая дубрированость документов.\\
3. Возможность обеспечивать унификации при конструировании, разработке, проэктированиии комерческих изделий\\
4. Упращенная форма конструкторской документации.\\
5. Механизм и автоматизм обработки технической документации.\\

\section{Методы проекции}

\subsection{Центральная проекция}

\subsection{Параллельная проекция и их свойства}

\subsection{Прямоугольное (ортогональное)проецирование}

\end{document}
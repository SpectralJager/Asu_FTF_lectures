\documentclass[a4paper, 12pt]{article}

\usepackage[table,xcdraw]{xcolor}
\usepackage[left=2.5cm, right=1.5cm, top=1.5cm, bottom=1.5cm]{geometry}
\usepackage{graphicx}
\usepackage{xcolor}
\usepackage{mdframed}
\usepackage { amsmath , amssymb , amsthm }
\usepackage[T2A]{fontenc}
\usepackage[utf8]{inputenc}
\usepackage[english,russian]{babel}
\usepackage{listings}
\usepackage{setspace}
\usepackage{amsmath}
\usepackage{float}
\usepackage{multirow}
\usepackage{lscape}


\onehalfspacing
\renewcommand{\familydefault}{\sfdefault}
% \renewcommand{\familydefault}{\sffamily}

\graphicspath{{img/}}
\DeclareGraphicsExtensions{.pdf,.png,.jpg}


\begin{document}
\begin{titlepage}
  \begin{center}
    \MakeUppercase{Министерство науки и высшего образования Российской Федерации} \\
    \MakeUppercase{ФГБОУ ВО Алтайский госудаственный университет}
    \vspace{0.25cm}
    
	  Институт цифровых технологий, электроники и физики
    
    Кафедра вычислительной техники и электроники
    \vfill
    
    {\LARGE Лабораторная работа №4. Теория двойственности}\\[5mm]
    \textsc{(Отчёт по лабораторным работам по курсу <<Методы оптимизации>>. \\13 вариант)}
  \bigskip

\end{center}
\vfill

\newlength{\ML}
\settowidth{\ML}{«\underline{\hspace{0.7cm}}» \underline{\hspace{1cm}}}
\hfill
\begin{minipage}{0.45\textwidth}
  Выполнил: ст. 595 гр.:\\
  \underline{\hspace{\ML}} Д.\,В.~Осипенко\\
  Проверил: к.ф-м. наук, доцент каф. ВТиЭ\\
  \underline{\hspace{\ML}} В.\,И.~Иордан\\
  «\underline{\hspace{0.7cm}}» \underline{\hspace{2cm}} \the\year~г.
\end{minipage}%
\vfill

\begin{center}
  Барнаул, \the\year~г.
\end{center}
\end{titlepage}

\newpage
\section{Краткие теоретические сведения}
Любой задаче линейного программирования можно поставить в соответствие другую задачу которая называется двойственной или сопряженной. Общие правила составления двойственных задач: 
\begin{enumerate}
  \item Во всех ограничениях задачи свободные члены должны находиться в правой части, а члены с неизвестными в левой. 
  \item Ограничения-неравенства исходной задачи должны быть записаны так, чтобы знаки неравенств были направлены в одну сторону. 
  \item Если знаки неравенств в исходной задаче «=<»то целевая функция должна максимизироваться, иначе минимизироваться.
  \item Каждому ограничению исходной задачи соответствует неизвестное двойственной задачи. 
  \item Целевая функция двойственной задачи должна оптимизироваться противоположным образом по сравнению с целевой функцией исходной задачи. 
\end{enumerate}
\section{Постановка задачи. 13 вариант}
Составить и решить двойственную задачу и, используя ее решение, найти решение исходной задачи
\begin{align*}
  Z(X)=x_1+3x_2+\frac{2}{3}x_3 \rightarrow \min\\
  \begin{cases}
    x_1-2x_2+x_3 \geq 2,\\
    3x_1+x_2+x_3 \geq 3,\\
    2x_1+3x_2-x_3\geq 1,\\
  \end{cases}
  \qquad x_j \geq 0, j=1,2,3
\end{align*}

Составляем двойственную задачу:
\begin{align*}
  F(Y)=-2y_1+3y_2+y_3 \rightarrow \max\\
  \begin{cases}
    y_1+3y_2+2y_3 \leq 1\\
    -2y_1+y_2+3y_3 \leq 3\\
    y_1+y_2-y_3 \leq \frac{2}{3}\\
  \end{cases}
\end{align*}

Введем дополнительные переменные:
\begin{align*}
  F(Y)=-2y_1+3y_2+y_3+0y_4+0y_5+0y_6 \rightarrow \max\\
  \begin{cases}
    y_1+3y_2+2y_3+y_4 = 1\\
    -2y_1+y_2+3y_3+y_5 = 3\\
    y_1+y_2-y_3+y6 = \frac{2}{3}\\
  \end{cases} 
\end{align*}

Найдем решение нашей задачи с помощью симплекс метода
\begin{table}[H]
\centering
\begin{tabular}{|c|c|c|c|c|c|c|c|c|}
\hline
   & Y1 & \cellcolor[HTML]{FFCCC9}Y2 & Y3 & Y4 & Y5 & Y6 & Решение & Отношение \\ \hline
\rowcolor[HTML]{FFFC9E} 
Y4 & 1  & 3                          & 2  & 1  & 0  & 0  & 1       & 1/3       \\ \hline
Y5 & -2 & \cellcolor[HTML]{FFCCC9}1  & 3  & 0  & 1  & 0  & 3       & 3         \\ \hline
Y6 & 1  & \cellcolor[HTML]{FFCCC9}1  & -1 & 0  & 0  & 1  & 2/3     & 2/3       \\ \hline
Q  & 2 & \cellcolor[HTML]{FFCCC9}-3  & -1  & 0  & 0  & 0  & 0       &           \\ \hline
\end{tabular}
\end{table}

Результат 1 итерации
\begin{table}[H]
\centering
\begin{tabular}{|c|c|c|c|c|c|c|c|}
\hline
   & Y1   & Y2 & Y3   & Y4   & Y5 & Y6 & Решение \\ \hline
Y2 & 1/3  & 1  & 2/3  & 1/3  & 0  & 0  & 1/3     \\ \hline
Y5 & -7/3 & 0  & 7/3  & -1/3 & 1  & 0  & 8/3     \\ \hline
Y6 & 2/3  & 0  & -5/3 & -1/3 & 0  & 1  & 1/3     \\ \hline
Q  & 3    & 0  & 1    & 1    & 0  & 0  & 1       \\ \hline
\end{tabular}
\end{table}

В строке Q отсутствуют отрицательные элементы, следовательно оптимальный план найден за 1 итерацию. Оптимальное решение двойственно задачи: 
\begin{align*}
  y_1=0,\quad y_2=\frac{1}{3},\quad y_3=0\\
  Y=(0,\frac{1}{3},0)\\
  \max F(Y) = -2\cdot0+3\cdot\frac{1}{3}+1\cdot0 = 1
\end{align*}
Найдем оптимальное решение исходной задачи по формуле: $X_{j\text{опт}}^{\text{пр}} = -Q_{m+j}^{\text{дв}}$
\begin{align}
  x_1=Q_{3+1}=Q_4=1\\
  x_2=Q_{3+2}=Q_5=0\\
  x_3=Q_{3+3}=Q_6=0\\
  X=(1;0;0)\\
  \min Z(x)=1\cdot 1+3\cdot0+\frac{2}{3}\cdot0 = 1
\end{align}
\section{Вывод}
Научились решить двойственные задачи симплекс методом. Нашли оптимальное решение двойственной и исходной задач. Решение двойственных задач применяется в экономическом анализе

\end{document}
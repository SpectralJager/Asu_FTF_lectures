\documentclass[a4paper, 12pt]{article}

\usepackage{graphicx}
\usepackage{xcolor}
\usepackage{mdframed}
\usepackage { amsmath , amssymb , amsthm }
\usepackage[T2A]{fontenc}
\usepackage[utf8]{inputenc}
\usepackage[english,russian]{babel}

\graphicspath{{img/}}
\DeclareGraphicsExtensions{.pdf,.png,.jpg}


\title{Экзамен}
\author{Осипенко}
\date{\today}

\begin{document}
\sffamily
\maketitle
\section{Основы теории множеств}
\subsection{Основные понятия теории множеств и способы их задания. Парадокс Рассела. Операции над множествами: объединение, пересечение, разность и симметрическая разность, дополнение. Свойства операций и принцип двойственности (правила Моргана)}
Обычно множества записываются в фигурных скобках.Множество может вообще не содержать ни одного элемента. В этом случае его именуют пустым множеством и обозначают как $ \emptyset $. Чаще всего в математической литературе множества обозначаются с помощью больших букв латинского алфавита.\\
Под мощностью множества для конечных множеств понимают количество элементов данного множества. Мощность множества A обозначается как |A|.\\
Если нам известно, что некий объект a принадлежит множеству A, то записывают это так: $ a \in A $.\\
Множество A называют подмножеством множества B, если все элементы множества A являются также элементами множества B. Обозначение: $ A \subseteq B $\\
Универсальное множество (универсум) U обладает тем свойством, что все иные множества, рассматриваемые в данной задаче, являются его подмножествами.\\
Множества A и B называются равными, если они состоят из одних и тех же элементов. Иными словами, если каждый элемент множества A является также элементом множества B, и каждый элемент множества B является также элементом множества A, то $ A = B $.\\
Если $ A \subseteq B $, при этом $ A \neq B $, то множество A называют собственным (строгим) подмножеством множества B. Также говорят, что множество A строго включено в множество B. Записывают это так: $ A \subset B $.\\
Множество всех подмножеств некоего множества A называют булеаном или степенью множества A. Обозначается булеан как P(A) или $ 2^A $. Пусть множество A содержит n элементов. Булеан множества A содержит $ 2^n $ элементов, т.е.
\[
       |P(A)| = 2^n, n = |A|
\]

Способы задания множеств:\\
1)Первый способ – это простое перечисление элементов множества. Естественно, такой способ подходит лишь для конечных множеств.\\
2)Второй способ – задать множество с помощью так называемого характеристического условия (характеристического предиката) P(x).
\[
       {x|P(X)}
\]
3)Третий способ – задать множество с помощью так называемой порождающей процедуры. Порождающая процедура описывает, как получить элементы множества из уже известных элементов или неких иных объектов.\\

Парадокс Рассела:\\ 
Пусть K — множество всех множеств, которые не содержат себя в качестве своего элемента. Содержит ли K само себя в качестве элемента? Если да, то, по определению K, оно не должно быть элементом K — противоречие. Если нет — то, по определению K, оно должно быть элементом K — вновь противоречие.(прим. Одному деревенскому брадобрею приказали «брить всякого, кто сам не бреется, и не брить того, кто сам бреется», как он должен поступить с собой?)\\

Симметрическая разность двух заданных множеств A и B - это такое множество $ A \bigtriangleup B $, куда входят все те элементы первого множества, которые не входят во второе множество, а, также те элементы второго множества, которые не входят в первое множество.\\

Законы де Моргана:\\
Отрицание конъюнкции есть дизъюнкция отрицаний.\\
Отрицание дизъюнкции есть конъюнкция отрицаний.\\

\subsection{Сравнение множеств. Диаграммы Эйлера-Венна. Разбиения и покрытия: принцип Гейне-Бореля-Лебега – лемма «о конечном подпокрытии». Алгебра подмножеств: булеан и универсум, счетные множества и их свойства. Несчетные множества и множества «мощности континуума». Теорема Кантора.}

Леммой Гейне - Бореля: 
Из всякой бесконечной системы интервалов, покрывающей отрезок числовой прямой, можно выбрать конечную подсистему, также покрывающую этот отрезок.\\ 

Алгебра множеств - это непустая система подмножеств некоторого множества X, замкнутая относительно операций дополнения (разности) и объединения (суммы). \\

Счётное множество есть бесконечное множество, элементы которого возможно занумеровать натуральными числами.\\
Свойства:
-В предположении, что выполнена аксиома выбора, любое бесконечное множество содержит счётное подмножество.\\
-Непустое подмножество счётного множества не более, чем счётно.\\
-Не более, чем счётное объединение не более, чем счётных множеств само не более, чем счётно.\\
-Декартово произведение конечного числа не более, чем счётных множеств само не более, чем счётно.\\

Несчётное множество - бесконечное множество, не являющееся счётным. Континуум в теории множеств - мощность множества всех вещественных чисел.\\

Теорема Кантора — классическое утверждение теории множеств. Доказано Георгом Кантором в 1891 году. Утверждает, что любое множество A менее мощно, чем множество всех его подмножеств $ 2^A $. \\

\subsection{Отношения. Упорядоченные пары. Прямое произведение множеств, бинарные отношения (обратное, дополнение, тождественное, универсальное). Композиция и степень отношений, ядро отношения. Свойства отношений.}
Отношение - математическая структура, которая формально определяет свойства различных объектов и их взаимосвязи. Распространёнными примерами отношений в математике являются равенство (=), делимость, подобие, параллельность и многие другие.\\

Два элемента a и b называются упорядоченной парой, если указано, какой из этих элементов первый, какой второй, при этом $ ((a,b) = (c,d)) \Leftrightarrow (a=c \land b=d) $.\\

Прямое, или декартово произведение двух множеств - множество, элементами которого являются все возможные упорядоченные пары элементов исходных множеств.\\

Бинарное отношение - отношения между двумя множествами.\\
Обратное отношение[уточнить] (отношение, обратное к R) — это двухместное отношение, состоящее из пар элементов ( y , x ), полученных перестановкой пар элементов (x, y) данного отношения  R. Обозначается: $ R^{{-1}} $. Для данного отношения и обратного ему верно равенство: $  (R^{{-1}})^{{-1}}=R $.\\

\subsection{Функции: определения, инъекция, сюръекция, биекция. Композиция (суперпозиция или сложная функция), индуцированная функция.}

Функция (отображение) - в математике соответствие между элементами двух множеств, установленное по такому правилу, что каждому элементу первого множества соответствует один и только один элемент второго множества.\\

Отображение (функция) $F:X \to Y$ называется сюръективным (или сюръекцией, или отображением на $Y$), если каждый элемент множества $Y$ является образом хотя бы одного элемента множества $X$, то есть $\forall y \in Y \exists x \in X : y=F(x)$.\\

Отображение (функция) $F$ множества $X$ в множество $Y(F: X \to Y)$ называется инъекцией (или вложением, или взаимно однозначным отображением множества $X$ в множество $Y$), если разные элементы множества $X$ переводятся в разные элементы множества $Y$, $\forall x \in X \exists y \in Y : y=F(x)$\\

Биекция - это отображение (функция), которое является одновременно и сюръективным, и инъективным.\\

Композиция функций (или суперпозиция функций) - это применение одной функции к результату другой\\

\subsection{Отношения эквивалентности: классы эквивалентности и фактормножества. Ядро функции.}
Отношение эквивалентности - бинарное отношение между элементами данного множества, свойства которого сходны со свойствами отношения равенства (симметричности, рефлексивности и транзитивности). 

\subsection{ Отношения порядка: минимальные элементы, частичный и линейный порядок.}

Бинарное отношение R на множестве X называется отношением нестрогого частичного порядка (отношением порядка, отношением рефлексивного порядка), если имеют место симметричность, рефлексивность и транзитивность.

\subsection{Замыкание отношений: замыкание отношения относительно свойства, транзитивное и рефлексивное транзитивное замыкания. Алгоритм Уоршалла.}


\end{document}
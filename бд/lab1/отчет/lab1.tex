\documentclass[a4paper, 12pt]{article}

\usepackage[left=1.5cm, right=1.5cm, top=1.5cm, bottom=1.5cm]{geometry}
\usepackage{graphicx}
\usepackage{xcolor}
\usepackage{mdframed}
\usepackage { amsmath , amssymb , amsthm }
\usepackage[T2A]{fontenc}
\usepackage[utf8]{inputenc}
\usepackage[english,russian]{babel}
\usepackage{setspace}
% \usepackage{indentfirst} 
\singlespacing 

\graphicspath{{img/}}
\DeclareGraphicsExtensions{.pdf,.png,.jpg}


\begin{document}
\begin{titlepage}
  \begin{center}
    \MakeUppercase{Министерство науки и высшего образования Российской Федерации} \\
    \MakeUppercase{ФГБОУ ВО Алтайский госудаственный университет}
    \vspace{0.25cm}
    
	  Институт цифровых технологий, электроники и физики
    
    Кафедра вычислительной техники и электроники
    \vfill
    
    {\LARGE Libre Office}\\[5mm]
    \textsc{(Отчёт по индивидуальному заданию по курсу <<Базы Данных>>)}
  \bigskip

\end{center}
\vfill

\newlength{\ML}
\settowidth{\ML}{«\underline{\hspace{0.7cm}}» \underline{\hspace{1cm}}}
\hfill
\begin{minipage}{0.45\textwidth}
  Выполнил ст. 3-го курса, 595 гр.:\\
  \underline{\hspace{\ML}} Д.\,В.~Осипенко\\
  Проверил: преп. каф. ВТиЭ\\
  \underline{\hspace{\ML}} Я.\,С.~Сергеева\\
  «\underline{\hspace{0.7cm}}» \underline{\hspace{2cm}} \the\year~г.
\end{minipage}%
\vfill

\begin{center}
  Барнаул, \the\year~г.
\end{center}
\end{titlepage}
\tableofcontents
\newpage
\section{Формирование структуры таблицы}
\subsection{Задание}
 Сформируйте  структуру  таблицы  СТУДЕНТ  для  хранения  в  ней 
справочных сведений о студентах, обучающихся в вузе 
\subsection{Решение}
\includegraphics[width=\textwidth]{"1-1.png"}\\

\section{Ввод и редактирование данных в режиме таблицы }
\subsection{Задание}
Введите данные, представленные на рис. 2.1, в таблицу  СТУДЕНТ, 
созданную ранее. 
\subsection{Решение}
\includegraphics[width=\textwidth]{"1-2.png"}\\

\section{Разработка однотабличных пользовательских форм }
\subsection{Задание}
Создайте  однотабличную  пользовательскую  форму  для  ввода  и 
редактирования данных таблицы СТУДЕНТ как на рис. 3.7. 
\subsection{Решение}
\includegraphics[width=\textwidth]{"3-1.png"}\\

\section{Разработка детального отчета}
\subsection{Задание}
С помощью мастера создайте детальный отчет для вывода данных 
таблицы СТУДЕНТ. Вид отчета показан на рис. 4.1. 
\subsection{Решение}
\includegraphics[width=\textwidth]{"4-1.png"}\\

\section{Команды поиска, фильтрации и сортировки}
\subsection{Задание}
Для  данных  таблицы  СТУДЕНТ  в  режиме  формы  найти  одну  из 
записей,  в  режиме  таблицы  отсортировать  записи  по  возрастанию 
значений одного из полей и отфильтровать данные в соответствии с 
критерием отбора.
\subsection{Решение}
\includegraphics[width=\textwidth]{"5-1.png"}\\

\section{Формирование запросов}
\subsection{Задание}
Сформируйте запрос-выборку, позволяющий получить из таблицы 
СТУДЕНТ данные о студентах мужского пола, родившихся после 2005 г. 
\subsection{Решение}
\includegraphics[width=\textwidth]{"6-1.png"}\\

\section{Разработка информационно-логической модели и 
создание многотабличной базы данных }
\subsection{Задание}
Создайте  структуру  таблиц  СЕССИЯ  и  СТИПЕНДИЯ,  а  в  ранее 
созданной таблице СТУДЕНТ установите ключевое поле в соответствии с 
табл.  1.2,  1.3  и  1.4.  Заполните  вновь  созданные  таблицы  СЕССИЯ  и 
СТИПЕНДИЯ, как это показано на рис. 1.4 и 1.5. 
\subsection{Решение}
\includegraphics[width=\textwidth]{"8-1.png"}\\

\section{Установление связей между таблицами}
\subsection{Задание}
Используя возможности LibreOffice Base, установите связи между 
созданными таблицами СТУДЕНТ, СЕССИЯ и СТИПЕНДИЯ базы данных 
СЕССИЯ.  
\subsection{Решение}
\includegraphics[width=\textwidth]{"7-1.png"}\\

\section{Разработка многотабличной пользовательской формы 
для ввода данных}
\subsection{Задание}
В  рамках  поставленной  ранее  задачи  рассмотрите  возможности 
ввода  информации  в  таблицы  на  основе  использования  составной 
формы.  Допустим,  студент  Кревцов  с  номером  личного  дела  16993 
вовремя  не  сдавал  сессию  по  уважительной  причине,  и  запись  с 
результатами сдачи экзаменов в таблице СЕССИЯ отсутствует.  
 
Постройте  составную  форму  и  довведите  недостающую 
информацию в базу данных. 
\subsection{Решение}
\includegraphics[width=\textwidth]{"9-1.png"}\\

\section{Формирование запросов для многотабличной базы 
данных }
\subsection{Задание}
Постройте  запрос  «ПРОЕКТ  ПРИКАЗА»,  позволяющий  выводить 
фамилию, имя, отчество и номер группы студентов, которым может быть 
назначена  стипендия,  а  также  размер  назначаемой  стипендии  в 
процентах  от  базового  размера  стипендии.  Эти  данные  могут  быть 
использованы при создании проекта приказа назначения студентов на 
стипендию по результатам экзаменационной сессии. Информация для 
получения  таких  данных  содержится  в  трех  связанных  таблицах 
СТУДЕНТ, СЕССИЯ и СТИПЕНДИЯ базы данных СЕССИЯ. 

На  основе  «ПРОЕКТА  ПРИКАЗА»  постройте  запрос  «ПРОЕКТ 
ПЛАТЕЖНОЙ ВЕДОСОСТИ», где размер стипендии указывается в рублях, 
а базовая стипендия вводится с клавиатуры как параметр. 
\subsection{Решение}
\includegraphics[width=\textwidth]{"10-1.png"}\\

\section{Разработка многотабличной формы отчета вывода 
данных }
\subsection{Задание}
Постройте  отчет  "ПРОЕКТ  ПРИКАЗА",  основанный  на  сформи-
рованном  ранее запросе  "ПРОЕКТ ПРИКАЗА",  выбирающем  из  таблиц 
базы данных СТУДЕНТ, СЕССИЯ и СТИПЕНДИЯ информацию о студентах, 
которым  по  результатам  экзаменационной  сессии  назначается 
стипендия, и о размере стипендии в процентах от базовой стипендии. 

Постройте  отчет  "ПЛАТЕЖНАЯ  ВЕДОМОСТЬ",  основанный  на 
сформированном ранее запросе "ПРОЕКТ ПЛАТЕЖНОЙ ВЕДОМОСТИ ", где 
размер стипендии указывается в рублях, а базовая стипендия вводится с 
клавиатуры  как  параметр.  Вывод  отчета  осуществите  в  электронные 
таблицы. 
\subsection{Решение}
\includegraphics[width=\textwidth]{"11-1.png"}\\
\includegraphics[width=\textwidth]{"11-2.png"}\\

\end{document}
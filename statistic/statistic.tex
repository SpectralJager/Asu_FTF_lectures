\documentclass[a4paper, 12pt]{article}

\usepackage{graphicx}
\usepackage{xcolor}
\usepackage{mdframed}
\usepackage { amsmath , amssymb , amsthm }
\usepackage[T2A]{fontenc}
\usepackage[utf8]{inputenc}
\usepackage[english,russian]{babel}

\graphicspath{{img/}}
\DeclareGraphicsExtensions{.pdf,.png,.jpg}


\title{Мат Статистика и теория вероятности}
\author{Дронов С.В}
\date{\today}

\begin{document}
\sffamily
\maketitle
\section{Случайные события}
Множество всех элементарных исходов -- $ \Omega $\\
Событие - подмножетво $ \Omega $  \\
А -- событие, $ w \in A  \Rightarrow \quad w$  исход благоприятный для А\\
Если $ w \leftarrow A $ реализовался, $ w $ исход благоприятный для A, то событие проихошло\\
$ \Omega $  -- достоверное\\
A,B -- события $ \Rightarrow A\cup B $ -- объединенные события, происходящие если происходит хотябы одно их событий А или B\\
$ AB $  -- пересечение событий происходит если происходят как A так и B\\
$ AB = 0 \Rightarrow$  A и B несовместны\\
$ A\#B $ - разность событий, происходит если А происходит, а B не происходит
$ \overline{A} $ - событие, когда А не происходит 
\section{Классическая вероятность}
$ \Omega $  - конечное множество\\
Все $ \omega \in \Omega $  равновозможны\\
Все $ A \in \Omega $ - события\\
$ |A| $ -число элементов A\\
$ P(A) = \frac{|A|}{|\Omega|} $ -- вероятность события А\\

Свойства:\\
1)$ P(\Omega) = 1, p(0) = 0, \forall  0 \leq P(A) \leq 1 $ \\
2)$ AB = 0 \Rightarrow P(A\cup B) = P(A) + P(B) $ \\
3)$ A \in B \Rightarrow p(A) \leq P(B) $ \\

\section{Геометрическая вероятность}
$ \Omega \in R^n $  - множество ограниченное и измеримое\\
Все $ \omega \in \Omega $  -равновозможны\\
События -- измеримые подножества $ \Omega $ \\
$ P(A) = \frac{\mu(A)}{\mu(\Omega)} $\\
$ \mu  $   - мера\\
Свойства 1-3 выполнены\\
\section{Статистическая вероятность}
Пусть n - раз ставится  независимые эксперименты по наблюдению события А\\
$ k_n(A) $  событие проихошло\\
$ \mu_n(A) = \frac{k_n(A)}{n} $ - относительная частота А\\
$ P(A) = \lim_{n\to inf} \mu_n(A)  $ - вероятностное событие от А\\
Свойства 1-3 выполнены\\
\section{Аксиомотическое определение вероятности}
$ \Omega $  - произвольное множество\\
$ f $ - система подмножеств $ \Omega $ , объявляемых событиями\\
отображение $ p:f\to R^+ $ - вероятность, если верно:\\
p1) 	$ P(\Omega) = 1 $  \\
p2) $ A,B \in f \quad AB = 0 \Rightarrow P(A\cup B) = P(A) + P(B) $ \\
p3) $ \{A_n, n \in N\}\in f, \quad i \neq j \Rightarrow A_i A_j = 0 \quad P(\cup_{n\in N}A_n) = \sum_{n=1}^{}P(A_N)$ \\
\section{Простейшие следствия аксиом}
1) $ P(0) = 0 $ \\
2) $ P(\overline{A})+P(A) = 1 $\\ 
3) $ A \in B \Rightarrow P(A) \leq P(B) $\\ 
4) $ A \in B \Rightarrow P(B \#A) = P(B) - P(A) $ \\
5) $ P(A\cup B) = P(A) + P(B) - P(AB) $ - формула сложения вероятностей\\
\section{Аксиома непрерывности вероятности}
p4) $ \forall {B_n , n\in N } \in f: \forall n \quad B_n \in B_{n+1} \quad P(\cup_{n\in N}B_n)= \lim_{n\to int} P(B_n)  $\\
\begin{mdframed}[backgroundcolor=blue!20] 
        Теорема\\
        Пусть выполнена p2, тогда $ p3 \Leftrightarrow p4 $ 
     \end{mdframed}

p5) $ \forall {C_n , n\in N} \ in f \quad  \forall n \quad  C_{n_1} \in C_n \quad  \Rightarrow P(\cap_n C_n) = \lim_{n\to inf} P(C_n) $ 



























\end{document}
\documentclass[a4paper, 12pt]{article}

\usepackage{graphicx}
\usepackage{xcolor}
\usepackage{mdframed}
\usepackage{listings}
\usepackage { amsmath , amssymb , amsthm }
\usepackage[T2A]{fontenc}
\usepackage[utf8]{inputenc}
\usepackage[english,russian]{babel}
\usepackage{color}

\definecolor{dkgreen}{rgb}{0,0.6,0}
\definecolor{gray}{rgb}{0.5,0.5,0.5}
\definecolor{mauve}{rgb}{0.58,0,0.82}

\graphicspath{{img/}}
\DeclareGraphicsExtensions{.pdf,.png,.jpg}
\lstset{frame=tb,
  language=c++,
  aboveskip=3mm,
  belowskip=3mm,
  showstringspaces=false,
  columns=flexible,
  basicstyle={\small\ttfamily},
  numbers=none,
  numberstyle=\tiny\color{gray},
  keywordstyle=\color{blue},
  commentstyle=\color{dkgreen},
  stringstyle=\color{mauve},
  breaklines=true,
  breakatwhitespace=true,
  tabsize=4
}

\title{Дз5}
\author{ОСИПЕНКО Д. 595}
\date{\today}

\begin{document}
\sffamily
\maketitle
\section{}
\section{fibonacci-01.cpp}
\begin{lstlisting}
#include <iostream>
#include <iomanip>
#include <vector>
#include <string>

using namespace std;

string bin(int n) {
    string b;
    while (n != 0) {
        b = (n % 2 == 0 ? "0" : "1") + b;
        n /= 2;
    }
    return b;
}

int main(int argc, char* argv[]) {
    if (argc != 2) {
        cout << "Example: fib.exe 5";
        return 2;
    }
    int n = atoi(argv[1]);
    vector<int> fib = { 1,1 };
    for (int i = 2; i < n; i++) {
        fib.push_back(fib[i - 1] + fib[i - 2]);
    }
    cout << setw(to_string(fib[n - 1]).length() + 2) << "dec" << setw(to_string(fib[n - 1]).length() + 2) << "hex"
        << setw(to_string(fib[n - 1]).length() + 2) << "oct" << setw(bin(fib[n - 1]).length()) << "bin" << endl;
    for (auto i : fib) {
        cout << setw(to_string(fib[n - 1]).length() + 2) << i;
        cout << setw(to_string(fib[n - 1]).length() + 2) << hex << i;
        cout << setw(to_string(fib[n - 1]).length() + 2) << oct << i;
        cout << setw(bin(fib[n - 1]).length() + 2) << bin(i) << endl;
    }
    return 0;
}
\end{lstlisting}
\section{}
Сначало находим наименьший элемент массива и меняем его с первым, потом находим следующий наименьший и меняем со вторым и т.д..
\begin{lstlisting}
#include <iostream>
using namespace std;

int* sel_sort(int a[], int l, int r) {
    for (int i = l; i < r; i++) {
        int min = i;
        for (int j = i + 1; j <= r; j++) {
            if (a[j] < a[min]) {
                min = j;
                int temp = a[i];
                a[i] = a[min];
                a[min] = temp;
            }
        }
    }
    return a;
}

int main() {
    int* a = new int[6]{ 8, 1, 13, 55, 173, 178 };
    cout << "Before: ";
    for (int i = 0; i < 6;i++) {
        cout << a[i] << " ";
    }
    a = sel_sort(a, 0, 5);
    cout << "\nAfter: ";
    for (int i = 0; i < 6; i++) {
        cout << a[i] << " ";
    }
    return 0;
}
\end{lstlisting}
\section{}


\end{document}
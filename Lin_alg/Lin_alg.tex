\documentclass[a4paper, 12pt]{article}

\usepackage { amsmath , amssymb , amsthm }
\usepackage[T2A]{fontenc}
\usepackage[utf8]{inputenc}
\usepackage[english,russian]{babel}
\usepackage{framed}


\title{Алгебра и аналит геометрия }
\author{Журавлев Евгений Владимирович}
\date{\today}

\begin{document}
\maketitle
\section{Лекция 1}
\subsection{Векторы}
Вектор(геометрический)-- это отрезок у которого указано начало и конец.
Точка будет рассматриваться как вектор начало и конец которого совпадает, такой вектор называется нулевым($\vec{0}$). Для не нулевых векторов $  \vec{AB}$.\\
Векторы называются коллинеарными если они лежат на одной прямой или параллельных прямых. Коллинеарные векторы называются сонаправленными если они направленны в одну сторону и противоположно напрвленными иначе.\\
Сонаправленные векторы называются равными если их длины равны. Длиной вектора называется длина отрезка.\\
Противоположно направленные векторы называются противоположными если их длины равны \[
	|\vec{AB}| = |\vec{BA}|
\] 
\[
	\vec{a} = -\vec{b}		
\]
Длинна вектора -- $|\vec{a}|$\\
Сложение векторов: \\
-Правило Треугольника:...\\
-Правило Параллелограмма:...\\
-Правило Многоугольника:...\\
Свойства сложения векторов:\\
1.$ (\vec{a} + \vec{b})+\vec{c} = \vec{a} + (\vec{b}+\vec{c}) $ Называется ассоциативным\\
2.Существование Нуля $ \vec{a} + \vec{0} = \vec{a} = \vec{0} + \vec{a} $\\
3.Коммутотивность $ \vec{a} + \vec{b}= \vec{b} + \vec{a} $\\
4. $ \vec{a} + (-\vec{a}) = \vec{0} $\\
\subsection{Произведение вектора и действительного числа}
Произведение дейсвительного числа альфа ($ \alpha \in R $) и $ \vec{a} $ называется вектор, обозначаемый $ \alpha\vec{a} $ длинна которого равна $ |\alpha||\vec{a}| $ а направление определяется следующим образом:\\
1. Если альфа больше нуля то $ \vec{a} $ и $ \alpha\vec{a} $ сонаправленны\\
2. Если альфа меньше нуля то $ \vec{a} $ и $ \alpha\vec{a} $ противоположно направленны\\
3. Если альфа равно нулю то  $ \alpha\vec{a} $ нулевой\\
Свойства:\\
1.$\forall \vec{a}\vec{t} \quad\forall \alpha \in R \quad\alpha(\vec{a}+ \vec{t}) = \alpha \vec{a} + \alpha \vec{b}$\\
2.$  \forall \vec{a} \quad\forall \alpha p\in R  \quad(\alpha + p)\vec{a} = \vec{a} \alpha + \vec{a}p$\\
3. $  1 \vec{a} = \vec{a} $\\
4. $  \forall \vec{a} \quad\forall \alpha p \in R  \quad(\alpha p)\vec{a}= \alpha p\vec{a}$\\
5. $  -\vec{a} = -1\vec{a} $\\
\begin{framed}
Теорема: Ненулевые $ \vec{a} $ и $ \vec{b} $ коллинеарны когда и только тогда когда существует дествительное число $ \alpha $ такое что $ \vec{a} = \alpha\vec{b} $ ($\vec{a}|\vec{b} \Leftrightarrow \exists \alpha \vec{a} = \alpha\vec{b}$).
\end{framed}


\subsection*{Литература}
Задачи по линейной алгебре: матрецы определители Журавлев Е.В\\
!!!(для индивидуальных работ)Сборник типовых заданий и примеров по аналит геометрии Журавлев Е.В\\
Векторы Журавлев Е.В Мальцева Е.Ю\\
Лошкеева В.Д Мальцев Ю.М Высшая алгебра и аналит геометрия(изд алтгу 2000)\\
Курош А.Г Курс высшей алгебры\\
Фаддеев Д.К Лекции по алгебре\\
Проскуряков И.В Сборник задач по линейной алгебре\\
Задачи по высшей алгебре Фаддеев Д.К Саминцкий Д.С\\
!!Высшая математика в урпражнениях и задачах Домков П.Е Попов А.Г Кожевникова Т.Я\\
!!Погорелов А.В Аналитическая геометрия\\
Александров П.С Аналитическая геометрия\\
Геометрия. Учебник для 10 -11 классов Атанасян\\



\end{document}
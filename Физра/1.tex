\documentclass[a4paper, 12pt]{article}

\usepackage{graphicx}
\usepackage{xcolor}
\usepackage{mdframed}
\usepackage { amsmath , amssymb , amsthm }
\usepackage[T2A]{fontenc}
\usepackage[utf8]{inputenc}
\usepackage[english,russian]{babel}


\title{Физра}
\author{Лопатина О.А}
\date{\today}

\begin{document}
\sffamily
\maketitle

\section{Лекция  -- Вводная} 

Вопросы спец группы красноармейских 90а 2 эт 34 каб\\
Требования: посетить кажд занятие и активно поработать и подготовить доклад по опред вопросу (7-10 мин) и возможно контр раб\\
Справ приним с гор больницы 4 или здрав пункт (прутцкая 103)\\
За пропуски готовить темы по выбору учителя и пройти тестирования\\
В ситеме мудл курс Физическая культура (код слово "физкультура")\\
на мед осмотр взять: паспорт и ксерокопию (1стр и прописка) полис омс снилс прививочная карта или сертифекат амбулаторная карта результат флюраграфич обследования\\
страница кафедра физ воспитания: студентам - каф физ восп\\

\newpage
\section{Лекция  -- Физическая культура в профессиональной подготовке студентов и социокультурное развитие личности студента}

\quad\textbf{Физическая культура} -- это часть общей культуры общества, направленная на укрепления повышение уровня здоровья, всесторон развит физических способностей и использования их в общественной практике и повседневности.\\

Физическое воспитание -- педогагический процесс, вид воспитания, специфическим содержанием котором является обучение движения, воспитанием физических качеств, овладение специальными знаниями, в потребности занятиях физ упраднениями.\\

Физическое развитие -- процесс изменения и совершенствования морфологических и функциональных систем организма человека в течении его жизни.\\

Физическое совершенство -- общественное совершенство об определенной мере гармонического физического развития во всесторонней физической подготовленности человека.\\

\subsubsection*{Виды физической культуры:}
-- Базовая(образовательная) - фундаментальная часть физ культура которая включенная в рамки образования.\\
-- Спорт - это вид физ культуры, который осуществляет соревновательную деятельность и подготовку к ней основанной на использовании физических упражнейний для достижения спортивных результатов.\\
-- Туризм - это вид физ культуры, включающий в себя активные виды туризма(пеший, конный, т.д), носит не только оздоровительный но и профессиональный характер.\\
-- Профессионально-прикладная физ культура(ППФК) - планомерно организованный процесс для использованния физ культуры для развития физических навыков для освоения той или иной профессии. Основа для ППФК состовляет ППФК подготовка.\\
-- Оздоровительная реабилетоционная физ клултура - процесс специально направленного в качестве средств лечения или востонавления функций организма, нарушенных в следствии определенных обстаятельств.\\
-- Адаптивная физическая культура - для лиц с ограниченными возможностями.\\
-- Спортивно-ревбелитационная  физ культура - востонавление функциональных возможностей после протяженных или тежелых физических занятий.\\
-- Фоновые виды(рекреативная и гегиеническая)\\

\subsubsection*{Средства физ культуры:}
1. Физические уражнения -- существует физиологическая классификация физ упражнений которые объединены по физиологическим принципам.\\
2. Оздоровительные \\
3. Гигиенические факторы\\

Физические качества -- сила, выносливость, быстрота, ловкость и гибкость.\\
Группы физических упражнений бывают циклическими и ациклическими, статическими и динамическими.\\
К оздоровительным силам природы относятся - солнце,воздуч и вода.\\
Гигиенические факторы - режим труда и отдыха, санитарно гигиенические условия, и т.д.\\

Организационно правовые основы физической культуры:\\
1. Конституция РФ(ст 41.)\\
2. Федеральный закон о "Физической культуре и спорте"(04/12/2004)\\



\newpage
\section{Лекция.  }
Индивидуальные выступления


\section{Лекция -- Социально-биологические основы адаптации организма человека к физической и умственной деятельности, факторам среды обитания}

\subsection{Комплексное воздействие факторов среды на физическое развитие и жизнидеятельность человека}

\textbf{Физ равитие} -- это непрерывно происходящии биологические процессы. На каждом возростном этапе они характерезуются определенным комплексом связанных между собой и с внешней средой морфологичесских, функциональных, биохимических, психических и др свойств организма и обусловленны этим своеобразием запасом физических сил. На физ развитие оказ влияние множество внешних и внутренних факторов:\\
- Окружающая среда\\
- Наследственность\\
- Образ жизни\\
- Отсутствие болезний\\
....\\

При изучении физ развития необойтись без влияния природных факторов(воздух, атмосферное давление, температура, влажность, солнечная радиация, тд). Из внешней среды в организм человека поступают вещества необходимые для его жизни деятельности и развития, а также раздражители полезные и вредные.\\

Рост отражает особенности пластических процессов протекающих в организме человека что обуславлевает важность качественного питания. Билок и витамины группы Б и витамины А Д Е. Положительно влияет сон(7-8 часов). \\

\textbf{Гармон роста:} с 21:00(22:00) до 1:00(2:00) происходит самая существенная выробатка мелотонина.\\

Все факторы среды в природе воздействуют на организм одновременно причем не каждый сам по себе а как сложный взаимодействующий комплекс. При этом наблюдаетс яусиление или ослабление одного фактора под влиянием другого.\\

Вывод: здоровый организм постоянно обеспечивает оптимальное состояние жизненых систем в ответ на любые изменения окружающей среды(перепады температур, атмосферного давления, влажности воздуха и тд).\\

\textbf{Д.З:}
 Выучить основные понятия. Основные анатамы морфологических понятий(организм, клетка,хромосомы,цитоплазма,рибосомы,цитолема,метахондрии, ассемиляция, дессимиляция, ткань, эпителиальные ткани, соединительные ткани, мышечные ткани, нервная ткань, орган). Учебник -- ВПС Муллер Физическая культура 2012(тема 2, пункт 2.2)\\

\subsection{Влияние природно климатических факторов на здоровье человека}

С точки зрения медико-биологических позиций влияние социально-экологических факторов на развитие и адаптационные свойства организма можно расмотреть на след примере:\\
1) Процесс аксселерации -- ускореннное развитие органов или частей организма по сравнению с некой биологической нормой.\\
2) Нарушение био-ритмов.\\


\section{Лекция}












\newpage
\subsection*{Литература} 
Физическая культура Муллер Вядичкина 2019\\
Физическая культура учебное пособие Лопатина Белоуско \\ 





\end{document}
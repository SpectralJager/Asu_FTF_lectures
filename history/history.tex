\documentclass[a4paper, 12pt]{article}

\usepackage{graphicx}
\usepackage{xcolor}
\usepackage{mdframed}
\usepackage { amsmath , amssymb , amsthm }
\usepackage[T2A]{fontenc}
\usepackage[utf8]{inputenc}
\usepackage[english,russian]{babel}

\graphicspath{{img/}}
\DeclareGraphicsExtensions{.pdf,.png,.jpg}


\title{История}
\author{Сковородов Александар Васильевич}
\date{\today}

\begin{document}
\sffamily
\maketitle
\section*{Требования}
- 10 лекций (не обязательно)\\
315М -- где искать\\
Курсы: История России часть 2 19-20 век , Современная история России.
Требования для автомата:\\
1) Практики обязательно(пропуски отрабатывать в электронном виде)\\
2) Если пропусков больше 3 необходимо писать сочинение(за каждый пропуск)\\
3)
4) Доклад на индивидуальную тему\\
4.1) Структура выступления: 15-20 минут, доклад должен состоять из 3х элементов:\\
\quad 1. ВВодная чать(задачи, актуальность, источники(вики нельзя!!!))\\
\quad 2. Основная чать(раскрытие содержание работы)\\
\quad 3. Заключение(итог)\\
5) ПО итогам подготовить письменную работу по данной теме(реферат -- по тойже тебе что и доклад, но развернутый анализ теме, объем 15-20 листов текста(1.5 интервал, 14 шрифт, поля 2 см)) сдать после майских(11 -- 15 мая,) празников.\\

\newpage
\section*{Темы}
1) Теории происхождения Древнерусского государства(нормандская, антинормандская, современная)\\
2) Политический портрет Александра Невского(Значение принятия полит решений)\\
3) Доктрина "Москва - 3 Рим"(условия, содержания, значение)\\
4) Учереждение Патриаршества(раскрыть понятие, условия принятия, последствия)\\
5) Реформа Патриарха Никона(Церковный раскол)\\
6) Бунташный век(17 век)(причины, участники, последствия)\\
7) Евроепизация и модернизация при Петре 1\\
8) Просвещенный абсолютизм при Екатерине 2(понятия, особенности в России)\\
9) Политический портрет императора Павла\\
10) Участие России в Наполеоновских войнах(что такое, особое внимание Отечественной войне)\\
11) Востание деокабристов(причины, особенности, значение)\\
12) Участие России в Крымской войне(Восточной)(причины войны, поражения, последствия)\\
13) Эпоха великих реформ\\
14) Промышленный переворот(особенности в России)\\
15) Политический портрет Александра 3\\
16) Реформы Петра Сталыпина(задача реформ, содержание, итоги, включить блок о влиянии реформ в развитии Алтая)\\
17) Альтернативы политического развития в 1913 году\\
18) Белый и красный террор в периуды гражданской войны\\
19) Культурная революция в СССР(20 - 30 годы)(методы, задача, итоги)\\
20) Индустриализация в СССР\\
21) Источники победы СССР в ВОВ\\
22) Антигитлеровская коалиция(вклад, значения)\\
23) Фестиваль молодежы и студентов в Москве 1957 году\\
24) Освоение космоса, космическая гонка(50 - 60 годы)\\
25) Олимпиада 1980 года(условия протекания, влияние + холодная война)\\
26) Катастрофа в чернобыле(авария, влияние на развитие СССР)\\
27) Распад СССР(причины(по блокам),последствия(положительный,отрицательные))\\
28) Участие России с международным терроризмом(конец 20- нач 21 века)(понятия, степень участия)\\
29) Мировой финансовый кризис 2008 года(понятия, влияние на развитие России)\\
30) Укрепление Российской государственности в начале 21 века(меры)







































\end{document}
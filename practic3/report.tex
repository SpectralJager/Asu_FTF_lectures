\documentclass[a4paper, 12pt]{article} 

\usepackage[table,xcdraw]{xcolor}
\usepackage[left=3cm, right=2cm, top=2cm, bottom=2cm]{geometry}
\usepackage{graphicx}
\usepackage{xcolor}
\usepackage{mdframed}
\usepackage { amsmath , amssymb , amsthm }
\usepackage[T2A]{fontenc}
\usepackage[utf8]{inputenc}
\usepackage[english,russian]{babel}
\usepackage{listings}
\usepackage{setspace}
\usepackage{amsmath}
\usepackage{float} 
\usepackage{multirow}
\usepackage{lscape}

\onehalfspacing
\renewcommand{\familydefault}{\sfdefault}
% \renewcommand{\familydefault}{\sffamily}

\graphicspath{{img/}}
\DeclareGraphicsExtensions{.pdf,.png,.jpg}


\begin{document}
\begin{titlepage}
  \begin{center}
    \MakeUppercase{Министерство науки и высшего образования Российской Федерации} \\
    \MakeUppercase{ФГБОУ ВО Алтайский госудаственный университет}
    \vspace{0.25cm}
    
	  Институт цифровых технологий, электроники и физики
    
    Кафедра вычислительной техники и электроники
    \vfill
    
    {\LARGE Практика по получении профессиональных\\ умений и навыков}\\[5mm]
    \textsc{(Отчёт по производственно-эксплуатационной практике направления подготовки бакалавров <<Информатика и вычислительная техника>>)}
  \bigskip

\end{center}
\vfill

\newlength{\ML}
\settowidth{\ML}{«\underline{\hspace{0.7cm}}» \underline{\hspace{1cm}}}
\hfill
\begin{minipage}{0.45\textwidth}
  Выполнил: ст. 595 гр.:\\
  \underline{\hspace{\ML}} Д.\,В.~Осипенко\\
  Проверил: ст. преп. каф. ВТиЭ\\
  \underline{\hspace{\ML}} И.\,А.~Шмаков\\
  «\underline{\hspace{0.7cm}}» \underline{\hspace{2cm}} \the\year~г.
\end{minipage}%
\vfill

\begin{center}
  Барнаул, \the\year~г.
\end{center}
\end{titlepage}

\newpage

\tableofcontents 
\newpage

\setcounter{section}{-1}
\section{Введение}
Осуществлено прохождение производственно-эксплуатационной практике (далее - практика) направленния подготовки <<Информатика и вычислительная техника>> на базе кафедры Вычислительной техники и электроники (ВТиЭ) Алтайского государственного университета (АлтГУ).  
Периуд прохождения практики 4 недели: с 16 мая, по 11 июня 2022 года. за этот периуд мной произведено:
\begin{enumerate}
  \item Ознакомление с номативно-правовой базой, регламентирующей деятельность программиста на месте практики:
  \begin{enumerate}
    \item Должностная инструкция программиста
    \item Инструкция по охране труда для программиста
    \item Вводный инструктаж
    \item Инструктаж по технике безопасности
  \end{enumerate}
  \item Диагнотика компьтеров на работоспособность, выявление технических проблем.
  \item Установка и настройка программного обеспечения для операционных систем Windows XP, 10 и Linux Ubuntu.
  \item Разработка скриптов для bash и powershell
  \item Разработка мобильного приложения на тему <<Устные вычисления>>
  \item Руководство группой людей
\end{enumerate} 

\subsection*{Цель практики}
\begin{enumerate}
  \item Получить знания и навыки их практического применения в сфере системного администрирования и программирования
  \item Ознакомится со спецификой деятельности системного администратора
  \item Развитие лидерских качеств
\end{enumerate}

\subsection*{Задачи практики}
\begin{enumerate}
  \item  поиск и изучение руководств по инсталляции, настройке, наладке, использованию программно-аппаратного обеспечения вычислительной техники, информационных и автоматизированных систем;
  \item  освоение методик использования необходимого программного обеспечения;
  \item  проверка работоспособности типовых узлов и устройств;
  \item  использование программного обеспечения для решения практических задач, составление схем приема-передачи данных.
\end{enumerate}

\newpage

\section{Место прохождения практики}
 
АлтГУ образован в соответствии с постановлением Совета Министров СССР от 27.03.1973 № 179 и приказом Министра высшего и среднего специального образования РСФСР от 24 мая 1973 г. № 229 как Алтайский государственный университет.

4 декабря 2002 года Алтайский государственный университет внесен в Единый государственный реестр юридических лиц как государственное образовательное учреждение высшего профессионального образования «Алтайский государственный университет».

Приказом Министерства образования и науки Российской Федерации от 28 апреля 2011 г. № 1546 Государственное образовательное учреждение высшего профессионального образования «Алтайский государственный университет» переименовано в федеральное государственное бюджетное образовательное учреждение высшего профессионального образования «Алтайский государственный университет».

Приказом Министерства образования и науки Российской Федерации от 21 апреля 2016 г. № 457 федеральное государственное бюджетное образовательное учреждение высшего профессионального образования «Алтайский государственный университет» переименовано в федеральное государственное бюджетное образовательное учреждение высшего образования «Алтайский государственный университет». 

Институт цифровых технологий, электроники и физики – это современный научно-образовательный центр, реализующий подготовку квалифицированных специалистов по высокотехнологичным направлениям естественнонаучного профиля и информационно-коммуникационным технологиям.

Кафедра вычислительной техники и электроники была образована на физико-техническом факультете Алтайского государственного университета в 1996 году для организации подготовки студентов и аспирантов в сфере информационных технологий и систем, а также компьютерной электроники и инженерии.

Сегодня на кафедре ведется подготовка по направлению «Информатика и вычислительная техника» по всем трем уровням высшего образования: подготовка бакалавров, подготовка магистров и подготовка кадров высшей квалификации – аспирантов. 

\section{Нормативная база программиста АлтГУ}

\section{Диагностика компьютеров. Устранение найденныйх проблем.}
Диагностика компьютера направленна на выявление проблем в его работе. В нашем случае план тестирования следующий:
\begin{enumerate}
  \item Попытка запуска
  \item Проверка подключенных устройств в BIOS
  \item Проверка настроек BIOS
  \item Проверка работоспособности RAM с помощью программы Memory Test 86 (memtest86) с загрузочной флешки
  \item Проверка HDD на наличие битых секторов с помощью * с загрузочной флешки
\end{enumerate}

В общем случае работа была проведена с 10тью компьютерами:

\begin{itemize}
  \item Для двух ПК была произведена замена/установка блока питанию
  \item У 4 ПК была заменена батарейка BIOS
  \item Для 5 ПК устранена проблема с ошибкой объема памяти для дискетоприемника путем изменения значения в BIOS
  \item У 1 ПК устранена проблема c S.M.A.R.T. (self-monitoring, analysis and reporting technology) при загрузке компьютера путем отключения параметра в BIOS
  \item У 5 ПК установлено верное значение даты и времени
  \item Для каждого компьютера установленны RAM примерно на 1гб (в сумме)
  \item Успешно протестированно 9 ПК на наличие ошибок RAM, 10 на проблем с HDD
  \item У 1 ПК выявленны проблеммы с разъемом оперативной памяти материнской платы
  \item Обнаружены две неисправные плашки оперативной памяти
\end{itemize}

\section{Установка Linux Ubuntu. Установка программного обеспечения}
Необходимо установить Linux на один диск с Windows для запуска с помощью Dual Boot. Логично предположить, что производить установку для каждого ПК по отдельности долго и утомительно, поэтому был выбран один ПК, на нем установленна Linux, произведенна настройка, установка всего необходимого программного обеспечения и после этого дополнительно подключается hdd от другого ПК и производится клонирование данных. 

Установка Ubuntu linux производилась с помощью стандартного GUI установщика, добавлены дистрибутивы АлтГУ, осуществленно разбиение свободной части диска на три раздела: загрузка (boot), виртуальная память (swap), домашний раздел (root/home). После завершения установки был отредактирован и запущен bash скрипт для скачивания необходимых программных покетов. 

Для Windows XP была произведенна активация с помощью ключа. Установленны недостающие драйвера и необходимы программы.

\section{Написание PowerShell скрипта}
Было необходимо удалить мусорные программы, преустанновленные на Windows 10 (например Cartana, Xbox, ...). В виду множества данных программ и колличества компьютеров, самым простым и логичным способом является написание Powershell скрипта. Скрипт состоит из блока отключения некоторых служб, открепления программ из меню пуска, удаления программ. Так же скрипт увеличивает количество виртуальной памяти по формуле: RAM memory * 2

\section{Руководство группой людей}
Осуществление помощи в контроле, помощи и руководстве над первокурсниками во время прохождения теми ознакомительной практики.

\section{Выполнение индивидуального задания}

\end{document}
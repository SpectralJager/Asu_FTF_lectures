\documentclass[a4paper, 12pt]{article} 

\usepackage[table,xcdraw]{xcolor}
\usepackage[left=3cm, right=2cm, top=2cm, bottom=2cm]{geometry}
\usepackage{graphicx}
\usepackage{xcolor}
\usepackage{mdframed}
\usepackage { amsmath , amssymb , amsthm }
\usepackage[T2A]{fontenc}
\usepackage[utf8]{inputenc}
\usepackage[english,russian]{babel}
\usepackage{listings}
\usepackage{setspace}
\usepackage{amsmath}
\usepackage{float} 
\usepackage{multirow}
\usepackage{lscape}

\onehalfspacing
\renewcommand{\familydefault}{\sfdefault}
% \renewcommand{\familydefault}{\sffamily}

\graphicspath{{img/}}
\DeclareGraphicsExtensions{.pdf,.png,.jpg}


\begin{document}
\begin{titlepage}
  \begin{center}
    \MakeUppercase{Министерство науки и высшего образования Российской Федерации} \\
    \MakeUppercase{ФГБОУ ВО Алтайский госудаственный университет}
    \vspace{0.25cm}
    
	  Институт цифровых технологий, электроники и физики
    
    Кафедра вычислительной техники и электроники
    \vfill
    
    {\LARGE Практика по получении профессиональных\\ умений и навыков}\\[5mm]
    \textsc{(Отчёт по производственно-эксплуатационной практике направления подготовки бакалавров <<Информатика и вычислительная техника>>)}
  \bigskip

\end{center}
\vfill

\newlength{\ML}
\settowidth{\ML}{«\underline{\hspace{0.7cm}}» \underline{\hspace{1cm}}}
\hfill
\begin{minipage}{0.45\textwidth}
  Выполнил: ст. 595 гр.:\\
  \underline{\hspace{\ML}} Д.\,В.~Осипенко\\
  Проверил: ст. преп. каф. ВТиЭ\\
  \underline{\hspace{\ML}} И.\,А.~Шмаков\\
  «\underline{\hspace{0.7cm}}» \underline{\hspace{2cm}} \the\year~г.
\end{minipage}%
\vfill

\begin{center}
  Барнаул, \the\year~г.
\end{center}
\end{titlepage}

\newpage

\tableofcontents 
\newpage

\section{Введение}
Осуществлено прохождение производственно-эксплуатационной практике (далее - практика) направленния подготовки <<Информатика и вычислительная техника>> на базе кафедры Вычислительной техники и электроники (ВТиЭ) Алтайского государственного университета (АлтГУ).  
Периуд прохождения практики 4 недели: с 16 мая, по 11 июня 2022 года. за этот периуд мной произведено:
\begin{enumerate}
  \item Ознакомление с номативно-правовой базой, регламентирующей деятельность программиста на месте практики:
  \begin{enumerate}
    \item Должностная инструкция программиста
    \item Инструкция по охране труда для программиста
    \item Вводный инструктаж
    \item Инструктаж по технике безопасности
  \end{enumerate}
  \item Диагнотика компьтеров на работоспособность, выявление технических проблем.
  \item Установка и настройка программного обеспечения для операционных систем Windows XP, 10 и Linux Ubuntu.
  \item Разработка скриптов для bash и powershell
  \item Разработка мобильного приложения на тему <<Устные вычисления>>
  \item Руководство группой людей
\end{enumerate} 

\subsection*{Цель практики}
\begin{enumerate}
  \item Получить знания и навыки их практического применения в сфере системного администрирования и программирования
  \item Ознакомится со спецификой деятельности системного администратора
  \item Развитие лидерских качеств
\end{enumerate}

\subsection*{Задачи практики}
\begin{enumerate}
  \item  поиск и изучение руководств по инсталляции, настройке, наладке, использованию программно-аппаратного обеспечения вычислительной техники, информационных и автоматизированных систем;
  \item  освоение методик использования необходимого программного обеспечения;
  \item  проверка работоспособности типовых узлов и устройств;
  \item  использование программного обеспечения для решения практических задач, составление схем приема-передачи данных.
\end{enumerate}



\end{document}
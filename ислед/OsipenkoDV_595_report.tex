\documentclass[14pt, oneside]{altsu-report}


\title{Методология тестирования информационных систем на проникновение}
\author{Д.\,В.~Осипенко}
\groupnumber{595}
\GradebookNumber{}
\supervisor{И.\,А.~Шмаков}
\supervisordegree{к.ф.-м.н., ст. преп.}
\ministry{Министерство науки и высшего образования}
\country{Российской Федерации}
\fulluniversityname{ФГБОУ ВО Алтайский государственный университет}
\institute{Институт цифровых технологий, электроники и физики}
\department{Кафедра вычислительной техники и электроники}
\departmentchief{В.\,В.~Пашнев}
\departmentchiefdegree{к.ф.-м.н., доцент}
\shortdepartment{ВТиЭ}

\abstractRU{
    Целью научно-иследовательской работы является разработка методологии тестирования на проникновение для информационных систем и её реализации в виде программного продукта на основе клиент-серверной архитектуры.
    
    Данная работа посвящена базовым этапам и методам для иследования защищенности информационных систем. Приведенны и рассмотрены теоретические сведенья о концепциях информационной безопасности, видах иследуемых систем, методах и этапах тестирования. 
}
% \abstractEN{Большой текст на английском!}
\keysRU{информационные системы, тестирование на проникновение, информационная безопасность}
% \keysEN{computer simulation, distributed version control}

\date{\the\year}

\addbibresource{graduate-students.bib}

\begin{document}
\maketitle

\setcounter{page}{1}
\makeabstract
\tableofcontents

\chapter*{Введение}
\addcontentsline{toc}{chapter}{Введение}

\chapter{Глава 1}
\section{Раздел 1}
\section{Раздел 2}

\chapter{Глава 2}
\section{Раздел 1}
\section{Раздел 2}

\chapter*{Заключение}
\addcontentsline{toc}{chapter}{Заключение}

\begin{enumerate}
\item Пример ссылки на литературу~\cite{wikiRUBitbucket}.
\item Пример ссылки на литературу~\cite{wikiRUIdSoftware}.
\item Пример ссылки на литературу~\cite{wikiRUGitHub}.
\item Пример ссылки на литературу~\cite{wikiRUSQLite}.
\end{enumerate}

\newpage
\addcontentsline{toc}{chapter}{Список использованной литературы}
\printbibliography[title={Список использованной литературы}]

\newpage
\chapter*{Приложение}
\addcontentsline{toc}{chapter}{Приложение}

\begin{code}
\captionof{listing}{Пример программы вычисления числа $\pi$ на языке \textit{C} с использованием \textit{MPI} (пример из https://ru.wikipedia
.org/wiki/Message\_Passing\_Interface)}
\label{code:pi-example}
\inputminted[mathescape,linenos,frame=lines,breaklines]{C}{src/pi-mpi.c}
\end{code}

\end{document}


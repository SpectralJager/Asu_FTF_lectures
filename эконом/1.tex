\documentclass[a4paper, 12pt]{article}

\usepackage[left=3cm, right=1.5cm, top=2cm, bottom=2cm]{geometry}
\usepackage{graphicx}
\usepackage{xcolor}
\usepackage{mdframed}
\usepackage { amsmath , amssymb , amsthm }
\usepackage[T2A]{fontenc}
\usepackage[utf8]{inputenc}
\usepackage[english,russian]{babel}
\usepackage{listings}
\usepackage{setspace}
\usepackage{indentfirst} 
\singlespacing 

\lstset{language=SQL,
  basicstyle={\small\ttfamily},
  belowskip=3mm,
  breakatwhitespace=true,
  breaklines=true,
  classoffset=0,
  columns=flexible,
  commentstyle=\color{dkgreen},
  framexleftmargin=0.25em,
  frameshape={}{yy}{}{}, %To remove to vertical lines on left, set `frameshape={}{}{}{}`
  keywordstyle=\color{blue},
  numbers=none, %If you want line numbers, set `numbers=left`
  numberstyle=\tiny\color{gray},
  showstringspaces=false,
  stringstyle=\color{green},
  tabsize=3,
  xleftmargin =1em
}

\graphicspath{{img/}}
\DeclareGraphicsExtensions{.pdf,.png,.jpg}


\begin{document}
\begin{titlepage}
  \begin{center}
    \MakeUppercase{Министерство науки и высшего образования Российской Федерации} \\
    \MakeUppercase{ФГБОУ ВО Алтайский госудаственный университет}
    \vspace{0.25cm}
    
	  Институт цифровых технологий, электроники и физики
    
    Кафедра вычислительной техники и электроники
    \vfill
    
    {\LARGE Оппортунизм и развитие экономики Российской Федерации: взаимосвязь}\\[5mm]
    \textsc{(Эссе по курсу <<Экономика>>)}
  \bigskip

\end{center}
\vfill

\newlength{\ML}
\settowidth{\ML}{«\underline{\hspace{0.7cm}}» \underline{\hspace{1cm}}}
\hfill
\begin{minipage}{0.45\textwidth}
  Выполнил ст. 3-го курса, 595 гр.:\\
  \underline{\hspace{\ML}} Д.\,В.~Осипенко\\
  Проверил: доц. каф. менеджмента, организации бизнеса и инноваций:\\
  \underline{\hspace{\ML}} О.\,В.~Кузнецова\\
  «\underline{\hspace{0.7cm}}» \underline{\hspace{2cm}} \the\year~г.
\end{minipage}%
\vfill

\begin{center}
  Барнаул, \the\year~г.
\end{center}
\end{titlepage}
\tableofcontents
\newpage

\section{Эссе}
На развитие экономики Российской Федерации большое влияние оказывают предприятия и фирмы.
Те же, в свою очередь, зависимы от персонала, который обеспечивает эффективность функционирования и устойчивости в рыночной среде.
Из-за различных факторов, например таких, как неудовлетворенность оплатой труда или рабочими условиями, может возникать самая распространенная форма оппортунистического поведения у сотрудников - "отлынивание".

Под оппортунизмом в современной экономической теории понимают преследование собственных интересов/выгоды, в том числе прибегая к различным махинациям, обману, лжи, умалчиванию некоторых невыгодных фактов и т.д.
По мнению исследователей - оппортунистическое поведение является одной из главных проблем в российском менеджменте, которое приводит к значительным издержкам экономической системы.

Оппортунизм может возникнуть из-за противоположности интересов работодателя и наемного работника.
Цель первого - максимизация прибыли и снижение издержек, в то время как интерес работника - повышение заработной платы и минимизация трудовых условий.
Это различие приводит к преследованию собственных интересов и экономии рабочей силы в ущерб фирме, а негибкость систем оценки и оплаты труда, свойственная для многих фирм, приводит к снижению мотивации труда и позволяет работникам некачественно выполнять трудовые обязанности.

В современной фирме оппортунистическое поведение может проявляться на двух уровнях: 1 уровень - взаимодействие между собственником и менеджерами; 2 уровень - взаимодействие между менеджерами и наемными работниками.

Причиной оппортунизма топ-менеджмент является передача собственниками управления фирмой менеджерам и формирование в связи с этим асимметрии информации в пользу менеджера и ситуации бесконтрольности его действий.

Оппортунизм  наемных  работников,  как  правило,  является  ответной  реакцией  на  оппортунизм со стороны работодателя.
Работники рассматривают данное обстоятельство как форму покушения на их экономическую свободу и вынуждены принимать защитные меры.

Оппортунизм  на  российских  предприятиях  проявляется  в  различных  формах  и  имеет  прямую зависимость с оппортунизмом, проявляемым со стороны руководства фирмы.

В итоге - оппортунизм возникает в результате нарушения равновесного соответствия в трудовых отношениях и направлен на повышение уровня экономической свободы каждой из сторон. В результате это приводит к общему снижению уровня экономической свободы фирмы. Выигравших в этом процессе – нет, проиграли и работодатель, и сотрудники и, самое главное,- фирма. И как следствие - негативное влияние на рост экономики страны.

\newpage
\section{Список литературы}
\begin{enumerate}
	\item Оппортунизм [Электронный ресурс], \\URL - https://ru.wikipedia.org/wiki/Оппортунизм
	\item Что такое оппортунизм? [Электронный ресурс], \\URL - https://dic.academic.ru/dic.nsf/ruwiki/128779
	\item Бодров О.Г. - Современные особенности оппортунизма сотрудников в российской фирме [Электронный ресурс], \\URL - https://davaiknam.ru/text/sovremennie-osobennosti-opportunizma-sotrudnikov-v-rossijskoj
	\item И.А. Семенова, А.С. Гербулова - ПРИЧИНЫ И ПОСЛЕДСТВИЯ ОППОРТУНИСТИЧЕСКОГО ПОВЕДЕНИЯ ПЕРСОНАЛА ФИРМЫ  [Электронный ресурс], \\URL - https://journals.udsu.ru/econ-law/article/view/5773
\end{enumerate}

\end{document}
\documentclass[a4paper, 12pt]{article}

\usepackage{graphicx}
\usepackage{xcolor}
%\usepackage{mdframed}
\usepackage { amsmath , amssymb , amsthm }
\usepackage[T2A]{fontenc}
\usepackage[utf8]{inputenc}
\usepackage[english,russian]{babel}

\graphicspath{{img/}}
\DeclareGraphicsExtensions{.pdf,.png,.jpg}


\title{Электротехника}
\author{Матюшенко Ю.Я.}
\date{\today}

\begin{document}
\sffamily
\maketitle
\section{Введение}
Электротехника - отрасть науки и техники, связанная с преобразованием, передачей и применением электрической энергии в жизнедеятельности человека.\\

Электротезника изучает количественные и качественные стороны электроманниткных процессво в электрических цупях и электромагнитном поле\\ \\
(Повторить: электр ток, напряжение, мощность)\\ 

Электрическая цепь - совокупность устройств и обьектов, образующих путь для электрического тока, электромагнитные процессы в которых могут быть описаны с помощью понятий об электрическом токе, ЭДС и электрическом напряжении\\
\\
Классификация:\\
- назначению\\
- режиму работы\\
- наличию нелинейных элементов\\
- способу соединения элементов\\
- числу ИП\\
- роду тока\\

цепи переменного тока делятся на периодические и непериодические, периодические - на синусоидальные и несинусоидальные\\
импульсные цепи - в них формируются и действуют импульсные, длящиеся мыйд интервал времени напряжения и токи\\

Три группы устройств:\\
1) источники электрической энернии\\
2) потребители электрической энергии\\ 
3) вспомогательные элементы цепи\\

Первичные источники -- источники, в которых происходит преобрахование неэлектричкской энергии в электричеескую

Вторичные -- источники, у которых и на входе, и на выходе - электрическая энергия\\

Потребители преобразуют электроэнергию в другие виды энергии\\

Вспомогательные элементы цепи: трансформаторы, соединительные провода, коммутационная аппаратура, аппаратура защиты, измерительные приборы и т.д., без которых реальная электрическая цепь не работает\\

Пассивные элементы делятся на:\\
- резистивные\\
- индуктивные\\
- емкостные\\

Резистр - элемент, обладающий электрическим сопротивлением, использутся для ограничения тока или создания падения напряжения определенной величины\\
\[
	W = i^2rt = uit \quad p = dW/dt = i^2 r =ui
\]

удельное сопротивление:
\[
	r = \rho l/S
\]
\newpage
\subsection{Линейные электрические цепи постоянного тока}
Топология электрических цепей:\\
% найти картинку топологии
-Узел - место соединения трех и более ветвей\\
-Ветвь - участок электрической цепи с одним и тем же током, состоющий из последовательно соединенных элементов\\
-Контур - замкнутый путь, проходящий по нескольким ветвям и узлам так, что ни одна ветвь и ни один узел не повторяются\\

Независимый контур имеет хотя бы одну ветвь, которой нет в других контурах.\\

За положительное направление ЭДС принимаются направления движения положительных зарядов внутри источника\\
%найти схему положит. направления ЭДС


Первый закон Ома:\\
- сила тока на участке цепи прямо пропорциональна напряжению на концах этого усастка и обратно пропорциональна его сопротивлению
\[
	U_{12} = \phi_1 - \phi_2, \quad I = U_{12}/R \quad U_{12} = RI
\]

Первый закон Кирхгофа:\\
\[
	\sum_{k=1}^{n} I_k = 0, \quad I_1 - I_2 - I_3 = 0, \quad I_1 = I_2 + I_3 
\]
Алгебраическая сумма токов ветвей, сходящихся в узле элкетр. цепи, равна нулю, т.у. сума втекающих токов равна сумме вытекающих\\

закон является следствием закона сохранения элкетр. заряда, согласно которому в оюбом узле электр. цепи заряд одного знака не может ни накапливаться, ни убывать.\\

Второй закон Кирхгофа:\\
алгебраическая сумма напряжений всех участков замкнутого контура равна нулю\\
\[
	\sum_{k=1}^{m} U_k = 0 
\]
где m - число участков контура\\

частный случай для схем с источниками ЭДС:
\[
	\sum_{k=1}^{m} U_{R_k} = \sum_{k=1}^{m} R_KI_k = \sum_{k=1}^{n} E_k   
\]

Закон Ома для полной(замкнутой) цепи:
\[
	I = E/(R_0 + R_E), \quad  U_{R_E} = E - IR_0
\]

Обобщенный закон Ома:
\[
	a)U_{34} = U_3 + E_1, \quad U_{34} = I_3R_3 + E_2, \quad I_3 = (U_{34} - E_2)/R_3
\]
\[
	b)U_{34} - U_3 = E_2, \quad U_{34} = Y_3 + E_2
\]
Сила тока в ветви с источником ЭДС прямо пропорциональна алгебраической сумме ЭДС и напряжения на ветви и обратно пропорциональна сопротивлению ветви\\
Чтобы сразу записать ответ, в выражении для тока со знаком "+" берут напряжение и ЖДС, направления которых совпадают с направленем тока.








\end{document}
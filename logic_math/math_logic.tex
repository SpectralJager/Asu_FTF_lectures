\documentclass[a4paper, 12pt]{article}

\usepackage{graphicx}
\usepackage{xcolor}
\usepackage{mdframed}
\usepackage { amsmath , amssymb , amsthm }
\usepackage[T2A]{fontenc}
\usepackage[utf8]{inputenc}
\usepackage[english,russian]{babel}

\graphicspath{{img/}}
\DeclareGraphicsExtensions{.pdf,.png,.jpg}


\title{Математическая логика и теория алгоритмов}
\author{Иордан В.И}
\date{\today}

\begin{document}
\sffamily
\maketitle
\section*{Введение в дисциплину}
Наука о законах и формах познающего мышления, она изучает мыслительные процессы направленные на обноружение и обоснование истин, на поиск путей преодоление трудностей в решении задач в различных сферах деятельности человека(в интелектуальной деятеьлности человека). Логику интересует лишь форма и законы мышления а не содержание(в меньшей степени), тем самым логика в своем развитии в большей степени интересуется грамматикой языковых выражений и проверяет их на истинность, другими словами наиболее важной задачей перед логикой стоит разработка формальных языков(формальных теорий) структуры которых должна быть максимально приближина к структуре естественных языков общения(к логике общения). Примерами являются алгоритмические языки программирования. Перед разработчиками таких новых алгоритмических языков программирования также стоит задача приближения этих новых языков программирования к уровню естественного языка программирования(с более высоким логическим уровнем). 

Математическая логика - это логика развиваемая с помощью математических методов, т.е логика используемая в математике как для развития самой логики(ее формализации), так и для развития других разделов математики. Она занимается построение новых формальных языков и формальных теорий, которые в последнее время развили и обогатили такие фундаментальные понятия как: отношение(функция), аксиома, теорема(доказательство), и д.р. Кроме того мат. логика в качестве новых формальных теорий языков изучает существующие логические исчисления и создает новые исчисления. В последнее время мат. логика обогатила понятия алгоритм, вычислимую функцию, построила система логического вывода(связанную с ИИ).
\newpage

Любая формальная теория состоит из 4 компонентов:\\
1) алфавит языка\\
2) формулы языка исчисления\\
3) аксиомы языка\\
4) правила вывода(построение новых формул из уже имеющихся)\\

1 и 2 компоненты образуют сигнатуру языка(формальный язык)\\

\section{Логика(алгебра) высказываний. Исчисление высказываний(ИВ)}
\subsection{}
Исчисления высказываний это состовная часть алгебры высказываний. ИВ опирается на аксиомотический подход в котором имеется ситема аксиом и правила вывода.

Высказывание - это утверждение, языковое предложение(не только человеком, но и как конструкция какого-либо алг. языка или предложения сформулированное ИИ), которое может быть либо истинной, либо ложью. Высказывания выраженные в виде формул используют обозначения переменных(латинский алфавит) и обозначения логических связок(операций). Части высказывания могут быть элементальными высказываниями(буквы) либо само исходное высказывание может быть элементарным.

Пропозициональная переменная(символ занимает определенную позицию)=переменная.

Отрицанием формулы А быдем называть новое высказываение С которое истинно тогда и только тогда, когда А - ложно.

Конъюнкцией высказываний А и Б будем называть новое высказываение С, которое истинно тогда и только тогда, когда истинны А и Б.

Дизъюнкцией высказываний А и Б будем называть новое высказывание С, которое истинно тогда, когда истинно А или Б.

Искл ИЛИ верно тогда и только тогда, когда верно только одно высказывание.

Импликацией высказываний А(условие, поссылка) и Б(заключение, следствие) быдем называть высказывание С, которое ложно тогда и только тогда, когда А истинно, а Б - ложно.

Эквиваленцией(эквивалентностью) высказываний А и Б будем называть высказывание С, которое истинно тогда оба высказывания одновременно истинны, либо ложны.

\subsection{Формулы ИВ}
Формулой будем называть составное высказывание, которое использую алфавит переменных строится согласно правилам синтаксиса:\\
1) Любая переменная -- суть формулы(тривиальный случай)\\
2) Если А и Б -- формулы, то формулами будут также инверсия и импликация.\\
3) формулы строются ТОЛЬКО по правилам 1 и 2.\\

Замечания: правила конструирования формул называют построение формул по "индукции" или по структуре формул.\\

Подформулой Б формулы А называют любое высказывание, входящие в составное высказывание А и являющиеся формулой.\\

Интерпретацией формулы А является определение значений истинности А при подстановки в нее конкретных истиностных значений переменных.\\

Формула истинная при некоторой интерпретации называется Выполнимой и она называется Опровержимой если она ложна при некоторой интерпритации.








\end{document}
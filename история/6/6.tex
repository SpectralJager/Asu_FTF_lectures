\documentclass[a4paper, 12pt]{article}

\usepackage{graphicx}
\usepackage{xcolor}
\usepackage{mdframed}
\usepackage { amsmath , amssymb , amsthm }
\usepackage[T2A]{fontenc}
\usepackage[utf8]{inputenc}
\usepackage[english,russian]{babel}

\graphicspath{{img/}}
\DeclareGraphicsExtensions{.pdf,.png,.jpg}


\title{Концепция «нового политического мышления» М.С. Горбачева и влияние ее реализации на международные процессы}
\author{Осипенко Д. 595гр.}
\date{\today}

\begin{document}
\sffamily
\maketitle
К 1987 г. сформировалась внешнеполитическая концепция, получившая название «новое политическое мышление». Она предполагала отказ от противостояния двух систем, признавала целостность и неделимость мира, объявляла приоритет общечеловеческих ценностей над классовыми и идеологическими в области международных отношений.

В следствии чего начался процесс нормализации отношений между США, Западом и Востоком. Подписаны договоры об уничтожении ракет средней и ближней дальности, о сокращении и ограничении стратегических наступательных вооружений, ликвидация части обычного вооружения и сокращения военного присутсвия в Европе.

Новое политическое мышление подразумевало отказ от прежнего курса в отношении социалистических стран. В 1989 г. начался форсированный вывод советских войск из стран Варшавского договора, что привело к росту антисоциалистических тенденций. Вскоре в ходе выборов и бархатных революций в этих странах произошла смена бывшего коммунистического руководства, и они стали ориентироваться на Запад, была объединена ГДР и ФРГ, ликвидированы ОВД и СЭВ.

Новое политическое мышление в области внешней политики дало противоречивые результаты. С одной стороны, ослабла гонка вооружений и угроза ядерной войны. Начался реальный процесс сокращения и уничтожения обычных и ядерных вооружений, наступило прекращение «холодной войны». Произошли демократические изменения в целом ряде стран. С другой стороны, такая политика привела к поражению СССР в «холодной войне», ликвидации всей мировой системы социализма и распаду биполярной системы международных отношений, существовавших на протяжении почти всего XX в.


\end{document}
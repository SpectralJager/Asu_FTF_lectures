\documentclass[a4paper, 12pt]{article}

\usepackage{graphicx}
\usepackage{xcolor}
\usepackage{mdframed}
\usepackage { amsmath , amssymb , amsthm }
\usepackage[T2A]{fontenc}
\usepackage[utf8]{inputenc}
\usepackage[english,russian]{babel}

\graphicspath{{img/}}
\DeclareGraphicsExtensions{.pdf,.png,.jpg}


\title{Становление новой российской государственности: 1992–2016}
\author{Осипенко Д. 595гр.}
\date{\today}

\begin{document}
\sffamily
\maketitle
Ликвидация СССР в 1991 г. резко усилила политическую борьбу в стране и отразились на взаимоотношениях исполнительной (Президент РФ и Правительство РФ) и законодательной (Верховный Совет и Съезд народных депутатов) ветвей власти. Лидеры «непримиримой оппозиции» вывели своих вооруженных сторонников на баррикады. Президент ввел в Москву войска. Это противостояние привело к кровопролитию. Оплот оппозиции – здание Верховного Совета было расстреляно из танков и взято штурмом. Лидеров оппозиции арестовали. 

12 декабря 1993 г. происходило всенародное голосование по принятию новой Конституции Российской Федерации, в результате которого она была одобрена большинством голосов и вступила в силу. Она подвела черту под советским периодом в российской истории.Была закреплена федеративная форма государства, разделение ветвей власти на законодательную, исполнительную и судебную, разнообразие видов собственности, включая частную собственность, широкие права и свободы граждан.  Высшим органом законодательной власти становилось Федеральное Собрание, состоящее из двух палат: Совета Федерации и Государственной Думы.

Российская Федерация добилась подтверждения своего международного статуса как правопреемника СССР, постоянного члена Совета Безопасности ООН, крупной ядерной державы.Тогда же определились главные направления российской внешней политики: отношения со странами ближнего зарубежья (бывшими республиками СССР, ставшими суверенными независимыми государствами) и отношения с ведущими мировыми державами (США, страны Западной Европы, Япония, Китай и др.).
\end{document}
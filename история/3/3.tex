\documentclass[a4paper, 12pt]{article}

\usepackage{graphicx}
\usepackage{xcolor}
\usepackage{mdframed}
\usepackage { amsmath , amssymb , amsthm }
\usepackage[T2A]{fontenc}
\usepackage[utf8]{inputenc}
\usepackage[english,russian]{babel}

\graphicspath{{img/}}
\DeclareGraphicsExtensions{.pdf,.png,.jpg}


\title{СССР в середине 1950-х – первой половине 1960-х гг.: попытка реформирования тоталитарного режима. Изменения в общественно-политической жизни общества}
\author{Осипенко Д. 595гр.}
\date{\today}

\begin{document}
\sffamily
\maketitle
После смери И.В. Сталина начались критика культа личности, освобождение и частичная ребилитация советских заключенных. За период с 1956 по 1961 гг. было реабилитировано 700 тыс. человек. Выступление Н. С. Хрущева на закрытом заседании ХХ съезда КПСС «О культе личности и его последствиях», а также принятие специального постановления ЦК КПСС от 30 июня 1956 г. положили начало критике сталинского режима. Выдвигалась задача «восстановления ленинских норм» в деятельности государства и КПСС.

В экономической сфере был проведен ряд реформ. С середины 1950-х гг. началась компания по освоению целинны. В 1957 году были введены территориальные советы народных хозяйств. Происходят преобразования колхозов в совхозы.В промышленной сфере нарастало отставание от ведущих западных стран. Рост промышленного и сельскохозяйственного производства с каждой пятилеткой постепенно снижался. Колхозники получили впервые паспорта, шло массовое строительство жилья.

Реформы Н. С. Хрущева не затрагивали основ командно-административной системы. В итоге прогрессивные начинания обернулись недовольством населения и партийно-государственного аппарата. В 1964 г. Н. С. Хрущев был освобожден от обязанностей.


\end{document}
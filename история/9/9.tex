\documentclass[a4paper, 12pt]{article}

\usepackage{graphicx}
\usepackage{xcolor}
\usepackage{mdframed}
\usepackage { amsmath , amssymb , amsthm }
\usepackage[T2A]{fontenc}
\usepackage[utf8]{inputenc}
\usepackage[english,russian]{babel}

\graphicspath{{img/}}
\DeclareGraphicsExtensions{.pdf,.png,.jpg}


\title{Либерализация экономической, политической и духовной жизни российского общества в 1990-е гг}
\author{Осипенко Д. 595гр.}
\date{\today}

\begin{document}
\sffamily
\maketitle

Кардинальные экономические преобразования российское правительство начало проводить с января 1992 г. При этом приоритетными направлениями реформы стали либерализация цен, приватизация государственной собственности, конверсия военно-промышленного комплекса, демонополизация производства. В аграрном секторе был взят курс на акционирование колхозов и развитие фермерского хозяйства.

Падение «железного занавеса» изменило и культурную жизнь общества. Граждане России открыли для себя западную культуру, смогли познакомиться с ранее запрещенными произведениями искусства, получили возможность свободного передвижения за рубеж. После десятилетий гонений государство коренным образом изменило свою политику по отношению к церкви. В настоящее время наблюдаются попытки возвести православие в ранг государственной религии вместе с констатацией важной роли ислама, буддизма и других традиционных для России конфессий.

Мо­но­поль­ная власть од­ной пар­тии сме­ни­лась мно­го­пар­тий­ной по­ли­ти­че­ской сис­те­мой, вы­бо­ры пред­ста­ви­тель­ной вла­сти при­ня­ли де­мо­кра­ти­че­скую фор­му.За­ко­но­да­тель­ны­ми ор­га­на­ми Рос­сии про­де­ла­на зна­чи­тель­ная ра­бо­та по фор­ми­ро­ва­нию но­во­го пра­во­во­го про­стран­ст­ва. При­ня­ты сот­ни за­ко­нов, вклю­чая це­лый ряд ко­дек­сов.

\end{document}
\documentclass[a4paper, 12pt]{article}

\usepackage{graphicx}
\usepackage{xcolor}
\usepackage{mdframed}
\usepackage { amsmath , amssymb , amsthm }
\usepackage[T2A]{fontenc}
\usepackage[utf8]{inputenc}
\usepackage[english,russian]{babel}

\graphicspath{{img/}}
\DeclareGraphicsExtensions{.pdf,.png,.jpg}


\title{Причины краха экономической реформы 1965 г.}
\author{Осипенко Д. 595гр.}
\date{\today}

\begin{document}
\sffamily
\maketitle
Осуществление экономической реформы 1965 г., называемой иногда «косыгинской реформой», началось с перехода к новой административной централизации, упразднения совнархозов и восстановления центральных промышленных министерств, ликвидированных Н.С. Хрущевым. Осуществление реформы дало стимул к развитию экономики, однако по мнению современных экономистов реформа была обречена на неуспех из-за целого ряда причин. Наиболее существенными из них были:

– непоследовательность и половинчатость, содержавшиеся непосредственно в самом замысле реформы. Сочетание экономических начал с жестко централизованной плановой экономикой, как показывает мировой и отечественный опыт, дает лишь кратковременный эффект, а затем вновь происходит доминирование административных принципов и подавление экономических;

– некомплексный характер реформы. Ни о какой демократизации производственных отношений, изменении форм собственности и перестройке политической системы речь даже не шла;

– слабая кадровая подготовленность и обеспеченность реформы. Инерция мышления руководящих хозяйственных кадров, давление на них прежних стереотипов, отсутствие творческой смелости и инициативы у непосредственных исполнителей преобразований обусловливали половинчатость замысла реформы и обрекали ее в итоге на неудачу;

– противодействие реформе со стороны партийного аппарата и его руководителей (Л.И. Брежнева, Н.В. Подгорного, Ю.В. Андропова), боявшихся, что экономика может выйти из-под партийного контроля, а реформа – поставить под сомнение сущность социалистического строя;

– чехословацкие события 1968 г., где аналогичные новации привели к началу демонтажа политической системы, что очень испугало советское руководство.

Экономическая реформа, будучи непоследовательной уже на этапе замысла, не была осуществлена должным образом. Она не смогла переломить неблагоприятные тенденции в экономическом развитии страны, а усилия партийного аппарата свели ее на нет. Вместе с тем реформа 1965 г. показала пределы и ограниченность социалистического реформаторства.

\end{document}
\documentclass[a4paper, 12pt]{article}

\usepackage{graphicx}
\usepackage{xcolor}
\usepackage{mdframed}
\usepackage{url}
\usepackage { amsmath , amssymb , amsthm }
\usepackage[T2A]{fontenc}
\usepackage[utf8]{inputenc}
\usepackage[english,russian]{babel}

\graphicspath{{img/}}
\DeclareGraphicsExtensions{.pdf,.png,.jpg}


\title{Доктрина "Москва - Третий Рим"}
\author{Осипенко Данил}
\date{\today}

\begin{document}
\sffamily
\maketitle
\section{Вводная часть}
"Москва -- третий Рим", теория, впервые возникшая в конце 15 - во второй половине 16 веков.\\

К основным задачам теории можно отнести:\\
- Объединение православных государств под знаменем Московского княжетсва\\
- Защита православной веры и уничтожение неверных\\
- Объявление Москвы центром православной веры\\
- Претензии на Византийское наследство\\

/В наше время теория имеет большой отпечаток на некоторых гражданах, в их самовосприятии и противопоставлении чужеродным телам./

\section{Основная часть}
Традиционно считается, что концепция "Москва -- третий Рим" принадлежит перу настоятеля псковского Елеазарова монастыря старцу Филофею. Он выразил её в двух посланиях. Одно из них адресовано великому князю Василию Третьему , другое -- дьяку Мунехину.Обстоятельства, при которых написаны эти послания, следующие. В январе 1510 года Псков был лишён своей автономии, вечевой колокол снят, ввели московскую систему управления. Руководить городом стали два московских наместника с двумя дьяками (одним из них и был Мунехин). Наместники творили "многия беззакония", и псковичи обратились за помощью к Филофею, лично знакомому с великим князем. Тот призвал их к терпению, но одновременно обратился с посланием к Василию, которое передал через отъезжающего в Москву Мунехина. В этом послании, как считается, Филофей косвенно изложил жалобы псковичей, но основными его темами были:  долгое отсутствие архиепископа Новгорода ("вдовство" епархии), неправильность совершения некоторыми людьми крестного знамения и "содомский грех". То, что потом получило название "теория "Москва -- третий Рим", высказано им в следующей мысле:

"Тебе пресветлейшему и выскопочтенному государю великому князю, православному христианскому царю и всея владыке святого Божего престола святого вселенского собора апостольской церкви Успения Пресвятой Богородицы. Старого Рима церковь пала от аполинариевой ереси, второго Рима Константинова града церковь от рук агарян(османов), над третим же, новым Римом, державы твоего царствия святые соборные апостольские церкви, во всей поднебесной ярче солнца светятся. Да и вся твоя держава, благочестивый царь, как все царства православной христианской веры сольюца в твое единое царство. Един ты во всей поднебесной христианам царь".

Здесь же содержится и знаменитая формула: "Два убо Рима падоша, третий стоить, а четвертому не быти." В послании дьяку он говорит почти то же самое, только там добавлена тема осуждения астрологов ("звездосчётцев").

Сложно в этих словах увидеть какую-либо законченную концепцию, тем более государственно-мессианского характера. Автор вкладывает в них отнюдь не имперский смысл: речь идёт о том, что Василий как единственный оставшийся православный монарх должен следить за чистотой православия и быть защитником веры. Именно так и было понято послание Филофея в XVI веке.

Как Г.В. Вернадский, так и Н.И. Ульянов считали, что широкого распространения концепция "Москва -- третий Рим" в это время не получила, оставаясь принадлежностью узкого круга монахов-книжников. Эту точку зрения разделял и М.А. Алпатов. Об этой теории в своих письмах не упоминает Иван Грозный, её нет ни в "официозных" летописных сводах (типа Воскресенского и Никоновского), ни в Степенной книге. Почти дословное изложение формулы Филофея появляется только в "утверждённой грамоте" константинопольского патриарха Иеремии, приехавшего в Москву в 1589 году для утверждения московской патриархии. Это подтверждает, что концепция была принадлежностью идеологии церковной, но не государственной.

И рождена теория нового Рима, сменяющего Константинополь, как показывает Н.И. Ульянов, не Филофеем. Видимо, впервые о " новом Царьграде" как о центре православного мира заговорили в Болгарии во временя расцвета Второго царства, когда Константинополь находился во власти крестоносцев. Под "новым Царьградом"  подразумевалось Тырново, где только что была утверждена патриархия. Затем, когда болгарское государство завоевали османы, беженцы, хлынувшие в Россию, занесли её в Москву, в помощи со стороны которой они видели единственную возможность свержения османского владычества.

Но в самой Москве к войне с Турцией отнеслись прохладно, вот почему теория "Москва -- третий Рим" не была принята на вооружение как часть государственной идеологии, оставшись внутри церкви. Настороженность к ней подпитывалась ещё и тем, что попытки втянуть Россию в войну с турками делались и с Запада. Добиться участия Москвы в антитурецких коалициях желали и Венеция, и Священная Римская империя германской нации, и папство. Ещё Ивана Третьего соблазняли "византийским наследством": " Восточная империя, захваченная оттоманами, должна, за прекращением императорского рода в мужском колене, принадлежать вашей сиятельной власти в силу вашего благополучного брака". При Василии Третьем Западом посылались посольства за посольством, московиты их вежливо выслушивали, но желания таскать для других каштаны из огня не проявляли. Послы уезжали, обижаясь на упрямство русских, неподдававшихся розовым миражам Константинополя. Программа, заложенная в умы московской элиты Иваном Третьим, исправно работала: бояре объясняли послам, что Россию интересует "киевское наследство".

Существование теории «Москва - Третий Рим» в качестве политической программы может быть поставлено под сомнение; «миф, что Москва стремилась представить себя третьим Римом, был создан позднейшими интерпретаторами и кажется намного более влиятельным, чем исконный миф о Москве – третьем Риме». Тема Послания Филофея великому князю Василию III – «не историческое место России как преемницы Византии, но дела Церкви». Что касается другого сочинения, излагающего эту доктрину, Послания Мунехину, то в нем Филофей «не триумфально провозглашает особую историческую роль России, как долгое время полагали ученые, но выражает свое видение опасной ситуации. Перенося это пророчество на современную ситуацию, Филофей заключает, что падение Римского имперского мира (the Roman empire would) привело к разрушению всего мира. Цель Филофеева сочинения – увещевательный: он никогда не намеревался легитимизировать притязания Москвы на роль третьего Рима, но пытался предостеречь о необходимости спасения последней христианской империи».

Контекст формулы «Москва – Третий Рим» у Филофея, действительно, богословский. По характеристике В.В. Колесова: «Главным сочинением Филофея является его послание псковскому дьяку Михаилу Григорьевичу Мунехину. Повод к написанию послания был следующий. Николай Булев, известный публицист, переводчик и врач при великом князе Василии III, любекский немец по происхождению, приблизительно в 1522 г. перевел астрологический «Альманах» Штоффлера, содержащий предсказание о потопе в 1524 г. Перевод попал в руки Федора Карпова и М. Г. Мисюря-Мунехина, каждый из них обратился за разъяснениями к знающим лицам. Для Федора Карпова таким близким корреспондентом оказался Максим Грек, для М. Г. Мисюря-Мунехина — старец Филофей. Оба ответа совпадают в своем отрицательном отношении к астрологии и датируются концом 1523 — началом 1524 г.

Рассуждение Филофея отвергают какое-либо значение астрологии, поскольку звезды как тела неодушевленные не могут оказывать влияния на судьбы людей или народов. Астрологии он противопоставляет иное объяснение исторического процесса: причиной изменений является божественная воля, причиной падения царств — неспособность удержаться в истинной вере. Эта историко-богословская концепция целиком находится в русле библейской историософии, но старцу Филофею необходимо примирить с нею падение православного Константинополя в 1453 г. и сохранение католическим Римом своего видимого благополучия. Вслед за этим он дает пространное обоснование подлинности причастия квасным (дрожжевым) хлебом, что позволяет ему объявить истинным Римом московскую Русь как единственно независимое и безупречное христианское государство. Признание за Римом первенствующего значения опирается на традиционную христианскую учение о Церкви.

Есть две возможности понимания смысла этой концепции. Во-первых, можно думать, с чем мы обычно и сталкиваемся, что послание Филофея дает политическое обоснование преемственности имперской власти от Рима к новому Риму — Константинополю — и далее к Москве. В этом случае мысль Филофея развивается параллельно или под влиянием так называемой концепции «переноса империи» , которая в условиях средневековой Европы давала обоснование для возведения новых европейских монархий в достоинство юридически правомочных наследников Римской империи. В нашем случае, однако, изложение политической идеи формулируется на типичном для московской публицистики языке богословия, хотя немаловажным моментом оказывается употребление старцем Филофеем терминов “царь” и “царство”. Напомним, что приобретение титула “царь” вместо прежнего “великий князь” стало позже одной из забот Ивана Грозного.

Другая трактовка послания не признает за ним политического значения. Так, Вл. Соловьев обратил внимание на то, что для Филофея не существует императорского Рима, но только папский, и это препятствует рассмотрению концепции в русле европейской модели «переноса империи», к тому же автор отмечает, что римская государственность сохраняет свое существование («ромейское царство неразрушимо»). По мнению Н. Ульянова, имперский мотив Москвы — третьего Рима уходит своими корнями не в XVI в., а в идейный и политический климат царствования Александра II, т. е. связан с “восточным вопросом” и развитием русского империализма.

Соловьевская оценка и точка зрения Н. Ульянова представляются предпочтительными. Правда, по мнению исследовательницы теории, ее контекста и метаморфоз в позднейшей русской культуре Н.В. Синицыной: «Мысль Филофея связана с идеей «переноса империи» двойным узлом: судьбы порабощенных христианских царств, которые “попрани от неверных”, теперь сосредоточены метаисторически в России; к ней переходит функция “неразрушимого” Ромейского царства. В пророчестве Филофея речь не могла идти о повторении, замещении, наследовании Греческого царства, неистребимые судьбы которого предрекал Псевдо-Мефодий. Наш пророк писал после его гибели и возвел Россию непосредственно к Ромейскому царству, которое было “старше” Византии и родилось с рождением христианства, чем и обеспечена гарантия его вечности, прочности его “стояния”. Оппозиция “Ромейское – Греческое" нужна Филофею, чтобы противопоставить разоренному второму неразрушимое первое и сделать его вместилищем Россию»

Почему же сейчас мы больше знаем о теории "Москва -- третий Рим"? Да потому что ей было суждено второе рождение в XIX веке. Если религиозной частью посланий Филофея еще пользовались в XVII веке раскольники, то потом забыли даже о ней. Забыли до тех пор, пока его послания не были опубликованы на рубеже 1850-1860-х годов в "Православном вестнике". И вот после этих публикаций и вошла в общественный дискурс его теория о третьем Риме. Этому способствовала как общественная, так и международно-политическая обстановка: споры между западниками и славянофилами, балканская проблема. Несколько строчек Филофея обросли пышной легендой. Вот этот-то образ и сохранился до наших дней.

\section{Итог}
Концепция Филофея отнюдь не стройная историософская схема, а довольно слабо структурированная и плохо отрефлектированная квази-схема. Падение первого Рима понимается в ней как отступление от веры, однако оно не приводит к завоеванию города/царства; напротив, падение второго Рима понимается как отступление от истинной веры в Послании Мунехину и как завоевание иноверцами-мусульманами в Послании великому князю, но при этом в первом тексте признается сохранение насельниками второго Рима православия (то есть отступничество оказывается временным, но карается более строго, чем отход от православия латинян). Эта размытость и позволяла ей быть столь долговечной, наполняться самым разным историософским содержанием в позднейшие столетия. В деградированном, утерявшим религиозный смысл виде она выродилась в комплекс национальной гордыни и исключительности, спародированный в одном из романов Виктора Пелевина: «Мы же и есть третий Рим. Который, что характерно, на Волге. Так что и в поход ходить никуда не надо. Отсюда наша полная историческая самодостаточноть и нацииональное достоинство»

\section{Источники}
- \url{http://www.odnako.org/blogs/moskva-v-xvi-veke-tretiy-rim-i-drugie-predstavleniya-o-sebe-v-mire-i-istorii/}\\
- \url{https://www.portal-slovo.ru/philology/44938.php}\\
- \url{https://w.histrf.ru/articles/article/show/moskva_trietii_rim}\\
- \url{https://bigenc.ru/domestic_history/text/2232667}\\
\end{document}
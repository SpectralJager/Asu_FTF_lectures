\documentclass[a4paper, 12pt]{article}

\usepackage{graphicx}
\usepackage{xcolor}
\usepackage{mdframed}
\usepackage { amsmath , amssymb , amsthm }
\usepackage[T2A]{fontenc}
\usepackage[utf8]{inputenc}
\usepackage[english,russian]{babel}

\graphicspath{{img/}}
\DeclareGraphicsExtensions{.pdf,.png,.jpg}


\title{Внешняя политика СССР в 1945 – начале 1950-х гг. Истоки «холодной» войны}
\author{Осипенко Д. 595гр}
\date{\today}

\begin{document}
\sffamily
\maketitle
Решающий вклад Советского Союза в победу антигитлеровской коалиции над фашизмом привел к серьезным изменениям на международной арене. СССР стали воспринемать как великую державу, было большое влияние на страны Восточной Европы и Китай, из-за чего в этих странах во второй половине 1940-х гг стали преобладать комунистические идеи.

 Двоих не может быть на вершине, Западная Европа во главе с США стала предринемать активные действия, для предотвращения палитического и идеологического "поглащения" СССР близлежайших стран. Манифестом противостояния стала речь У. Черчилля в г. Фултон 5 марта 1946 г., где он призвал западные страны бороться с «экспансией тоталитарного коммунизма». В Москве это выступление было воспринято как политический вызов. И.В. Сталин резко ответил У. Черчиллю в газете «Правда», отметив: «…что по сути дела г-н Черчилль стоит теперь на позиции поджигателей войны». Конфронтация еще больше усилилась, и «холодная война» разгорелась с обеих сторон. СССР проводил экономическую и военную поддержку странам Восточной Европу, в ответ на которую стала программа экономической помощи странам, пострадавшим от нацистской агрессии, провозглашенная 5 июня 1947 г. государственным секретарем США Дж. Маршаллом. Отказавшись от помощи США, страны Восточной Европы по инициативе СССР создали в январе 1949 г. собственную международную экономическую организацию – Совет Экономической Взаимопомощи (СЭВ).

В конце 1940-х – начале 1960-х гг. конфронтация между СССР и США усилилась в Европе и Азии. На Дальнем Востоке в 1950–1953 гг. разразилась Корейская война между Севером и Югом, которая стала практически открытым военным столкновением между противоборствующими блоками.


\end{document}
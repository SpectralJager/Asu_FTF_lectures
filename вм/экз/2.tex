\documentclass[a4paper, 12pt]{article}

\usepackage{graphicx}
\usepackage{xcolor}
\usepackage{mdframed}
\usepackage { amsmath , amssymb , amsthm }
\usepackage[T2A]{fontenc}
\usepackage[utf8]{inputenc}
\usepackage[english,russian]{babel}

\graphicspath{{img/}}
\DeclareGraphicsExtensions{.pdf,.png,.jpg}


\title{Экз}
\author{Osipenko}
\date{\today}

\begin{document}
\sffamily
\maketitle
\section{Многочлены Тейлора.}
Теорема Тейлора даёт приближение к функции, дифференцируемой k раз, вблизи данной точки с помощью многочлена Тейлора k-го порядка.
\[
       P_n(x) = f(x_0) + \frac{f'(x_0)}{1!}(x - x_0) + \ldots + \frac{f^{(n)}(x_0)}{n!}(x - x_0)^n = \sum_{k = 0}^{n} \frac{f^{(k)}(x_0)}{k!}(x - x_0)^k
\]
\section{Интерполяционный многочлен Лагранжа. Линейная интерполяция.}
Интерполяционный многочлен Лагранжа - многочлен минимальной степени, принимающий данные значения в данном наборе точек. Для n+1 пар чисел (x0, y0), (x1, y1),…, (xn, yn), где все xj различны, существует единственный многочлен L(x) степени не более n, для которого L(xj) = yj. 

В простейшем случае (n=1) — это линейный многочлен, график которого — прямая, проходящая через две заданные точки. 
\[
       L(x) = \sum_{i = 0}^n y_il_i(x), \quad l_i(x) = \prod_{j = 0,j\neq i}^n \frac{x - x_j}{x_i - x_j}
\]
\section{Минимизация оценки погрешности интерполяции по Лагранжу. Многочлены Чебы-шева}
Погрешность интерполяции полинома Лагранжа имеет вид:
\[
       |f(x) - L_n(x)| \leq \frac{M_{n+1}}{(n+1)!}\text{max}|(x-x_0)\ldots(x-x_n)|
\]
\[
    \text{max}|(x-x_0)\ldots(x-x_n)| = \text{max}|\omega_{n+1}(x)| \rightarrow \text{min}
\]

Многочлен Чебышева 
\[
       T_{n+1}(x) = \frac{(b-a)}{2^{2n+1}}\cos\left((n+1)\arccos\frac{2x-(b+a)}{(b-a)} \right) )
\]
\section{Интерполяция по Лагранжу с равноотстоящими узлами.}
\[
       x_1 - x_0 = \ldots = x_n - x_{n-1} = h, \quad \frac{x-x_0}{h} = t 
\]
\[
       (-1)^{n-1}C_n^l \frac{t(t-1)\ldots(t-n)}{(t-l)n!}
\]
\section{Интерполяционный многочлен Ньютона и разделенные разности.}
Разделенные разности нулевого порядка совпадают со значениями функции в узлах. Разделенные разности первого порядка определяются через разделенные разности нулевого порядка: 
\[
       f(x_i,x_{i+1}) = \frac{f(x_{i+1}) - f(x_i)}{x_{i+1}-x_i}
\]
Разделенные разности второго порядка определяются через разделенные разности первого порядка. Разделенные разности k-го порядка определяются через разделенные разности порядка $ k-1 $:
\[
        f(x_i,\ldots,x_{i+k}) = \frac{f(x_{i+1},\ldots,x_{i+k}) - f(x_i,\ldots,x_{i+k-1})}{x_{i+k}-x_i}
\]
Используя понятие разделенной разности интерполяционный многочлен Ньютона можно записать в следующем виде: 
\[
       P_n(x) = f(x_0) + f(x_0,x_1)(x - x_0) + \ldots + f(x_0,\ldots,x_n)(x-x_0)(x-x_1)\ldots(x-x_n-1)
\]
\section{Численное дифференцирование.}
Численное дифференцирование - совокупность методов вычисления значения производной дискретно заданной функции. В основе численного дифференцирования лежит аппроксимация функции, от которой берется производная, интерполяционным многочленом. Все основные формулы численного дифференцирования могут быть получены при помощи первого интерполяционного многочлена Ньютона.

Один из универсальных способов построения формул численного дифференцирования состоит в том, что по значениям функции f(x) в некоторых узлах $ x_{0},x_{1},\ldots ,x_{N} $ строят интерполяционный полином $ P_{N}(x) $(в форме Лагранжа или в форме Ньютона) и приближенно полагают
\[
       f^{{(r)}}(x)\approx P_{N}^{{(r)}}(x),0\leq r\leq N
\]

В ряде случаев, наряду с приближенным равенством удается (например, используя формулу Тейлора) получить точное равенство, содержащее остаточный член R (погрешность численного дифференцирования)
\[
       f^{{(r)}}(x)=P_{N}^{{(r)}}(x)+R,0\leq r\leq N
\]

Такие формулы называются формулами численного дифференцирования с остаточными членами. 
\section{Сплайны. «Дефекты» сплайнов. Теорема о погрешности приближения сплайном.}
Сплайн - функция в математике, область определения которой разбита на конечное число отрезков, на каждом из которых она совпадает с некоторым алгебраическим многочленом (полиномом). Максимальная из степеней использованных полиномов называется степенью сплайна. Разность между степенью сплайна и получившейся гладкостью (производной функции) называется дефектом сплайна. Например, непрерывная ломаная есть сплайн степени 1 и дефекта 1.

Другими словами сплайн - это кусочно заданная функция, то есть совокупность нескольких функций, каждая из которых задана на каком-то множестве значений аргумента, причём эти множества попарно непересекающиеся. 
\section{Равномерные приближения функций. Теоремы Чебышева.}
Чтобы многочлен $ Q_{n}(x)  $ степени n был многочленом наилучшего равномерного приближения непрерывной функцииf(x), необходимо и достаточно существования на [ a , b ] по крайней мере n+2 точек x0 < ... < xn + 1 таких, что
\[
       f(x_{i})-Q_{n}(x_{i})=\alpha (-1)^{i}||f-Q_{n}||
\]

где i = 0 , ... , n + 1 , $ \alpha $ = +- 1 одновременно для всех i.

Точки x0 < ... < xn + 1, удовлетворяющие условиям теоремы, называются точками чебышёвского альтернанса. 
\section{}
\section{}
\section{}
\section{}
\section{}
\section{}
\section{}
\section{}
\section{}
\section{}
\section{}
\section{}
\section{}
\section{}
\section{}
\section{}
\section{}
\section{}
\section{}
\section{}
\section{}
\section{}
\section{}
\section{}
\section{}
\section{}
\section{}
\section{}
\section{}
\section{}
\section{}
\section{}
\section{}
\section{}
\section{}
\section{}


\end{document}
\documentclass[a4paper, 12pt]{article}

\usepackage{graphicx}
\usepackage{xcolor}
\usepackage{mdframed}
\usepackage { amsmath , amssymb , amsthm }
\usepackage[T2A]{fontenc}
\usepackage[utf8]{inputenc}
\usepackage[english,russian]{babel}

\graphicspath{{img/}}
\DeclareGraphicsExtensions{.pdf,.png,.jpg}


\title{Информатика}
\author{Шмаков И. А.}
\date{\today}

\begin{document}
\maketitle
\section*{Лекция 1}
Информатика -- это наука изучающая информационные аспекты процессов и системные аспекты информационных процессов.\\
Термин впрвые появился в 1957 году благодаря Карлу Штейнбуху. В 1962 Дрейфусом во Франции. Харкувич в 1962 в СССР.\\
Объем данных -- кол-во байт, необходимых для их хранения в памяти электронного носителя. Бит -- базовая единица измерения кол-ва информации.\\
Машинное слово -- машино-зависящее и платформо-зависящее величина, измеряющаяся в битах или байтах.


\end{document}
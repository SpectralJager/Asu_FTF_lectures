\documentclass[a4paper, 12pt, twoside]{article}

\usepackage[left=3cm, right=1.5cm, top=2cm, bottom=2cm]{geometry}
\usepackage{graphicx}
\usepackage{xcolor}
\usepackage{mdframed}
\usepackage { amsmath , amssymb , amsthm }
\usepackage[T2A]{fontenc}
\usepackage[utf8]{inputenc}
\usepackage[english,russian]{babel}
\usepackage{listings}
\usepackage{setspace}
\usepackage{indentfirst} 
\usepackage{enumitem}
\usepackage[pagestyles]{titlesec}

% text parameters
\linespread{1.25}
\renewcommand{\familydefault}{\sfdefault}
\titlespacing*{\section}{1.25cm}{1.25cm}{1.25cm}
\titleformat{\section}{\normalfont\bf}{\thesection}{0.5cm}{}\title

\graphicspath{{img/}}
\DeclareGraphicsExtensions{.pdf,.png,.jpg}

\usepackage{fancyhdr}
\fancyhf{} % clear all header and footers
\renewcommand{\headrulewidth}{0pt} % remove the header rule
\fancyfoot[LE,RO]{\thepage} % Left side on Even pages; Right side on Odd pages
\pagestyle{fancy}
\fancypagestyle{plain}{%
  \fancyhf{}%
  \renewcommand{\headrulewidth}{0pt}%
  \fancyhf[lef,rof]{\thepage}%
}



\begin{document}
\begin{titlepage}
  \begin{center}
    \MakeUppercase{Министерство науки и высшего образования Российской Федерации} \\
    \MakeUppercase{ФГБОУ ВО Алтайский госудаственный университет}
    \vspace{0.25cm}
    
	  Институт цифровых технологий, электроники и физики
    
    Кафедра вычислительной техники и электроники
    \vfill
    
    {\LARGE Должностная инструкция для специалиста <<Системный-инженер>>}\\[5mm]
    \textsc{(Практическая работа №3 по курсу <<Организация производства и управление предприятием>>)}
  \bigskip

\end{center}
\vfill

\newlength{\ML}
\settowidth{\ML}{«\underline{\hspace{0.7cm}}» \underline{\hspace{1cm}}}
\hfill
\begin{minipage}{0.45\textwidth}
  Выполнил ст. 3-го курса, 595 гр.:\\
  \underline{\hspace{\ML}} Д.\,В.~Осипенко\\
  Проверил: ст. преп. каф. ВТиЭ:\\
  \underline{\hspace{\ML}} Н.\,Н.~Плотицын\\
  «\underline{\hspace{0.7cm}}» \underline{\hspace{2cm}} \the\year~г.
\end{minipage}%
\vfill

\begin{center}
  Барнаул, \the\year~г.
\end{center}
\end{titlepage}
\tableofcontents
\newpage

\setlist[enumerate]{topsep=0pt,itemsep=0ex,partopsep=0ex,parsep=0ex,labelsep=4ex,itemindent=7ex,leftmargin=0ex,wide=1.25cm}

\section{Общие положения}
\begin{enumerate}[label=1.\arabic*.]
  \item Настоящая Должностная инструкция определяет подчинение, нормативно-правовую основу
   деятельности, должностные обязанности, взаимодействие, права и ответственность, а также 
   требования к квалификации сервис-инженер (далее – Работник) ООО <<Рюмка водки на столе>> 
   (далее – Общество).
  \item Работник относится к категории специалистов.
  \item Работник непосредственно подчиняется старшему сервис-инженеру.
  \item Работник принимается на работу и увольняется генеральным директором
Общества в установленном действующим трудовым законодательством РФ порядке.
  \item В период отсутствия Работника его обязанности выполняет работник, назначенный приказом генерального директора Общества.
  \item На должность Работника назначается лицо, имеющее 
высшее образование или прошедшее программы обучения по повышению квалификации и стаж работы на должносте техник сервисной службы или кодировщик не менее 0.5 лет.
  \item Работник при выполнении должностных обязанностей руководствуется:
  \begin{enumerate}[label=-]
    \item Действующим законодательством РФ;
    \item Отраслевыми нормативными и методическими документами в области ОКЗ, ЕКС и ОКСО;
    \item Положением о ООО <<Рюмка водки на столе>> и настоящей Должностной инструкцией;
    \item Приказами и распоряжениями генерального директора Общества, указаниями непосредственного руководителя;
    \item Правилами внутреннего трудового распорядка Общества;
    \item Иными локальными нормативными актами Общества;
    \item Нормативными актами по противопожарной защиты.
  \end{enumerate}
  \item Работник должен знать:
  \begin{enumerate}[label=-]
    \item действующее законодательство Российской Федерации и отраслевые нормативные документы в области ОКЗ, ЕКС и ОКСО;
    \item согласование и утверждение требований к типовой информационной системе (далее – ИС), модульное тестирование ИС, создание пользовательской документации к модифицированным элементам типовой ИС, развертывание серверной части ИС у заказчика, мониторинг выполнения договоров на выполняемые работы, связанные с ИС;
    \item структуру Общества, задачи и функции его подразделений;
    \item основы эксплуатации компьютерной техники, коммуникаций и связи;
    \item правила и нормы охраны труда.
  \end{enumerate}
  \item Режим работы Работника определяется в соответствии с Правилами внутреннего трудового распорядка, установленными в Обществе.
  \item Настоящая Должностная инструкция, изменения и дополнения к ней утверждаются и вводятся в действие приказом генерального директора Общества.

\end{enumerate}

\section{Должностные обязанности}
\begin{enumerate}[label=2.\arabic*.]
  \item Работник выполняет следующие должностные обязанности:
  \begin{enumerate}[label=2.1.\arabic*.]
    \item Согласование требований к типовой ИС с заинтересованными сторонами, запрос дополнительной информации по требованиям к типовой ИС и утверждение требований к типовой ИС;
    \item Тестирование разрабатываемого модуля ИС и устранение обнаруженных несоответствий;
    \item Разработка частей руководства пользователя/администратора/программиста к модифицированным элементам типовой ИС;
    \item Проверка соответствия серверов требованиям ИС к оборудованию и программному обеспечению, установка и проверка правильности установки серверной части ИС у заказчика;
    \item Формальный контроль договорных обязательств по выполняемым работам по срокам поставок и платежам;
    \item Подготовка отчетности о статусе исполнения договоров на выполняемые работы; 
  \end{enumerate}
\end{enumerate}

\section{Права}
\section{Взаимодействие}
\section{Ответственность}



\end{document}
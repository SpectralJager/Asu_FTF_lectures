\documentclass[a4paper, 12pt]{article}

\usepackage{graphicx}
\usepackage{xcolor}
\usepackage{mdframed}
\usepackage { amsmath , amssymb , amsthm }
\usepackage[T2A]{fontenc}
\usepackage[utf8]{inputenc}
\usepackage[english,russian]{babel}

\graphicspath{{img/}}
\DeclareGraphicsExtensions{.pdf,.png,.jpg}


\title{Распад СССР и образование СНГ. Успехи, трудности, перспективы развития Содружества}
\author{Осипенко Д. 595гр.}
\date{\today}

\begin{document}
\sffamily
\maketitle
В условиях ослабления государственной машины вспыхнули тлевшие до этого времени межнациональные конфликты.Накал политической борьбы в обществе сопровождался ухудшением социального положения граждан.Всеобщий дефицит товаров, спад уровня жизни населения привели к началу широкого забастовочного движения по всей стране.Попытки летом 1990 г. провести в жизнь программу «500 дней», рассчитанную на поэтапный переход к рынку, провалились вследствие сопротивления консервативной части бюрократии.Таким образом, налицо был серьезный социально-политический кризис союзного государства. Уже с марта 1990 г. национальные республики одна за другой начали принимать декларации о государственном суверенитете.

Юридическим закреплением распада СССР стало подписание в Беловежской пуще в декабре 1991 г. договора между Россией, Украиной и Белоруссией о прекращении деятельности всех структур Советского Союза, там же было объявлено об образовании СНГ в качестве межгосударственного объединения трех стран, затем к СНГ присоединились еще 8 государств бывшего СССР, а в 1993 г. – Грузия. Таким образом, на постсоветском пространстве образовалось 15 независимых государств, 12 из которых (кроме прибалтийских стран) продолжали сотрудничать между собой в рамках СНГ.

Основными сдерживающими факторами развития можно назвать продолжающийся на всем пространстве СНГ спад производства; слабость экономик; проблема взаимных неплатежей предприятий; переориентация экспортных ресурсов на рынки дальнего зарубежья в целях получения твердой валюты; конкурентные преимущества товаров стран дальнего зарубежья не только по уровню качества и себестоимости, но и надежности торгового партнерства; сознательная диверсификация импорта странами СНГ с целью уменьшения импортной зависимости от России по стратегически важным товарам в пользу дальнего зарубежья; неработоспособность системы реальной конвертации национальных валют; отсутствие действенного механизма платежно-расчетных отношений; значительный рост транспортных тарифов, что делает в ряде случаев экономически нецелесообразным традиционный товарообмен. В итоге незрелость рыночных институтов, слабость в странах Содружества, в том числе и в России, среднего предпринимательского класса заставляет вести поиск новых принципов развития сотрудничества на макро- и микроуровне. 
\end{document}
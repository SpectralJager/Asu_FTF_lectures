\documentclass[a4paper, 12pt]{article}

\usepackage{graphicx}
\usepackage{xcolor}
\usepackage{mdframed}
\usepackage { amsmath , amssymb , amsthm }
\usepackage[T2A]{fontenc}
\usepackage[utf8]{inputenc}
\usepackage[english,russian]{babel}

\graphicspath{{img/}}
\DeclareGraphicsExtensions{.pdf,.png,.jpg}


\title{Особенности экономического и политического развития СССР в 1945-1953 гг}
\author{Осипенко Д.В 595гр.}
\date{\today}

\begin{document}
\sffamily
\maketitle
Военные годы обошлись СССР потерей трети национального богатства, и огромного кол-ва людских ресурсов, из-за чего стал выбор пути востановления экономики.\\
Было выбранно вернуться к модели 30-х годов, старт четвертого пятилетнего плана, начало коллективизации сельского хозяйства - создание совхозов. Источниками для восстановляния советской экономики стали немецкие репарации, денежная реформа 1947 г., бесплатный труд узников ГУЛАГа и военнопленных, стремление советских людей в строительстве мирной жизни.\\
Экономическими этогами можна назвать стремительный рост в 1947-48 гг. и последующее замедление, длившимся вплоть до 1954 года, становление сельского хозяйтсва самой отсталой отраслью.\\

После победы во Второй мировой войне за СССР был признан статус "Великой державы", число стран, установивших дипломатические отношения увеличилось в два раза.\\
Усилиями советских и американских дипломатов удалось создать такие основополагающие структуры политического и экономического порядка, как ООН, Международный валютный фонд, Всемирный банк.  Но счастье длилось не долго, из-за различных противоречий, например таких как разный политический строй, влияние в Европе и т.д., между США и СССР стали накаливатся отношения, в последствии ставшии причинами начала "холодной войны". В втоже время внутри самой СССР стал набирать свой пик культ личности Сталина, ужесточения контроля над жизнью общества.\\

Обобщая можно сказать, что данный периуд жизни СССР был довольно не однозначный, как в политическом, так и в экономическом плане, в каждом из котором взлеты сменяли падения.

\end{document}
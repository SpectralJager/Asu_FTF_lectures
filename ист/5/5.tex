\documentclass[a4paper, 12pt]{article}

\usepackage{graphicx}
\usepackage{xcolor}
\usepackage{mdframed}
\usepackage { amsmath , amssymb , amsthm }
\usepackage[T2A]{fontenc}
\usepackage[utf8]{inputenc}
\usepackage[english,russian]{babel}

\graphicspath{{img/}}
\DeclareGraphicsExtensions{.pdf,.png,.jpg}


\title{Анализ политического реформирования как элемента концепции перестройки}
\author{Осипенко Д. 595гр.}
\date{\today}

\begin{document}
\sffamily
\maketitle
Вступление СССР в эпоху радикальных преобразований относится к апрелю 1985 г. и связано с именем нового Генерального секретаря ЦК КПСС М. С. Горбачева (избранного на этот пост на мартовском пленуме ЦК).

На январском 1987 г. пленуме ЦК КПСС, а затем на XIX Всесоюзной партконференции (лето 1988 г.) М. С. Горбачевым была изложена новая идеология и стратегия реформ. Впервые признавалось наличие деформаций в политической системе и ставилась задача создания новой модели социализма – «социализма с человеческим лицом».

В идеологию «перестройки» были включены некоторые либерально-демократические принципы (разделения властей, представительной демократии (парламентаризма), защиты гражданских и политических прав человека). На XIX партконференции впервые была провозглашена цель – создание в СССР социалистического гражданского (правового) общества. Демократизация и гласность стали сущностными выражениями новой концепции социализма. Началось «переосмысление» истории, был принят указ о реабилитации жертв полит. репрессий и о возвращении советского гражданства всем, кто был его лишен с 1966 по 1988 г. . Была отменена 6-я статья Конституции, закреплявшая монополию на власть для КПСС, в следствии чего начили свое формирования политические партии и блоки. 

В итоге из-за политических преобразований и неоднозначности из оценок в обществе вызвали борьбу вокруг содержания, темпов и методов реформ, сопровождавшуюся все более острой борьбой за власть, разные социальные слои стали активно включатся в политическо-общественную жизнь. Тем самым КПСС связала себе руки и сделала невозможным возращение к старым порядкам.

\end{document}
\documentclass[a4paper, 12pt]{article}

\usepackage{graphicx}
\usepackage{xcolor}
\usepackage{mdframed}
\usepackage { amsmath , amssymb , amsthm }
\usepackage[T2A]{fontenc}
\usepackage[utf8]{inputenc}
\usepackage[english,russian]{babel}

\graphicspath{{img/}}
\DeclareGraphicsExtensions{.pdf,.png,.jpg}


\title{СССР на международной арене в середине 1950-х – первой половине 1960-х гг. Берлинский и Карибский кризисы}
\author{Осипенко Д. 595гр.}
\date{\today}

\begin{document}
\sffamily
\maketitle
Осознавая реальную угрозу ядерного оружия, Председатель Совета министров СССР Г. М. Маленков, а позже Н. С. Хрущев считали, что в ядерный век мирное сосуществование государств является единственно возможной основой межгосударственных отношений, в следствии чего на 20 съезде КПСС были утверждены тезисы:\\

–  о мирном сосуществовании как форме классовой борьбы,\\

–  о возможности предотвращения войны в современную эпоху,\\

–  о многообразии форм перехода различных стран к социализму.\\

В качестве главных направлений обеспечения мира Н. С. Хрущев назвал создание системы коллективной безопасности в Европе, а затем в Азии, а также достижение разоружения. В периуд с 1950 - 1960-х гг. СССР всячески проявлял миролюбивые инциативы, сокращала свои Вооруженные силы, ликвидировала военные базы, заключала договоры с Западными странами.

В то же время в советской внешнеполитической доктрине оставались серьезные противоречия, определявшиеся коммунистической идеологией. Современная эпоха характеризовалась КПСС как время перехода к социалистической революции. В рамках следования принципу пролетарского интернационализма ставилась задача оказания всемерной (в том числе военной и военно-технической) поддержки национально-освободительным движениям в странах «третьего мира», которые становились ареной борьбы двух сверхдержав.

В серьезный кризис вылились события 1961 г. в ГДР, где значительная часть населения выступала за изменение политического строя страны. В августе 1961 г. в ответ на массовое бегство восточных немцев в Западный Берлин между двумя частями города была воздвигнута Берлинская стена, ставшая символом противостояния Востока и Запада.

Летом 1962 г. по решению советского руководства, в целях обезопасить Кубу (после того как весной 1961 г. США пытались свергнуть правительство Ф. Кастро) и изменить в свою пользу военно-политический баланс на континенте, на острове тайно были размещены советские ядерные ракеты средней дальности. Обноружив их, США привела войска в полную боевую готовность, тоже сделало и СССР. В течение 22–27 октября Дж. Кеннеди и Н. С. Хрущеву удалось прийти к заключению временного компромисса: СССР согласился демонтировать и вывезти с Кубы все ракеты, США, в свою очередь, гарантировали безопасность Кубы, а также согласились вывезти ракеты с военных баз в Турции и Италии.

В итоге можно сказать, что международная обстановка характеризовалась определенной стабилизацией и снижением международной напряженности, Советская внешняя политика претерпела изменения в сторону либерализации курса, при этом неизменным оставался курс на непримиримое противоборство с мировым капитализмом, сохранялся примат идеологии над политикой, что приводило к острейшим политическим кризисам на международной арене.
\end{document}
\documentclass[a4paper, 12pt]{article}

\usepackage{graphicx}
\usepackage{xcolor}
\usepackage{mdframed}
\usepackage { amsmath , amssymb , amsthm }
\usepackage[T2A]{fontenc}
\usepackage[utf8]{inputenc}
\usepackage[english,russian]{babel}

\graphicspath{{img/}}
\DeclareGraphicsExtensions{.pdf,.png,.jpg}


\title{Принципиальные отличия Конституции РФ 1993 г. от Конституции СССР 1977 г}
\author{Осипенко Д. 595гр}
\date{\today}

\begin{document}
\sffamily
\maketitle
Так как СССР являлась социалистической, а РФ демократической страной, то соответственно само содержание двух разных Конституций имела отличия. Конституция РФ 1993 года закрепляет новые нормы о принятии новой Конституции (в частности, Конституционным Собранием). Отменено право приостановления отдельных статей Основного закона законодательным органом. Ликвидация системы Советов, так же как и Съезд народных депутатов, прошла ликвидация в законодательстве о закреплении земли и в качестве достояния народностей определенной территории. Должным образом изменена должность Вице-президента Российской Федерации. Изменена регламентация структуры представительных и исполнительных органов местного самоуправления, как в прежнем Основном законе статус местных Советов народных депутатов и местной администрации входил в регламент непосредственно. В высшем нормативно-правовом акте 1993 года принципы избирательной системы выведены в законы. Конституционными федеральными законами устанавливается символика государства.

Подводя итог, нужно отметить, что при переходе нашей Великой страны к демократическому режиму, всё что было в Конституции 1977 года было заменено как частично, так и полностью. Вся суть заключается в том, что переход к демократии от социализма всё - таки повлияло не только на финансовое состояние, интересы людей и прочее, но и то, что пришло новое время – время идти дальше. А значит и законодательство не должно оставаться на прежнем месте.
\end{document}
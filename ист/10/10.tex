\documentclass[a4paper, 12pt]{article}

\usepackage{graphicx}
\usepackage{xcolor}
\usepackage{mdframed}
\usepackage { amsmath , amssymb , amsthm }
\usepackage[T2A]{fontenc}
\usepackage[utf8]{inputenc}
\usepackage[english,russian]{babel}

\graphicspath{{img/}}
\DeclareGraphicsExtensions{.pdf,.png,.jpg}


\title{Внешняя политика России в 2000-2016 гг}
\author{Осипенко Д. 595гр.}
\date{\today}

\begin{document}
\sffamily
\maketitle
В 2000-е годы внешнеполитическая стратегия РФ существенно изменилась. Россия отказалась от практики односторонних уступок, которую демонстрировала в предыдущее десятилетие. На первый план были четко и однозначно поставлены интересы России, которые и стали определять действия российской дипломатии. В основу внешней политики были положены выдвинутые Президентом доктрина национальной безопасности и доктрина информационной безопасности России.Были восстановлены отношения с НАТО, после событий 11 сентября 2001 г. в США, правительство РФ содействовало реализации мер по борьбе с террористической угрозой, предпринятых этой страной. Россия открывает воздушный коридор для переброски грузов для американских войск в Афганистане. Россия установила отношения с такими государствами, как Индия, Китай, Бразилия, ЮАР, вошедшими в образованную РФ – БРИКС. Первостепенное значение отводится экономическому сотрудничеству. В 2014 г. создается Евразийский экономический союз. Также в этом году было осуществленно принятие Крыма и Севастополя в состав РФ, которое было негативно воспринято странами запада и влечет последствия вплоть по сегодняшний день.


\end{document}